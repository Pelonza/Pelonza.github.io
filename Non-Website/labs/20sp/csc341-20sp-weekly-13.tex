\documentclass[12pt]{article}

%%%%% Document Information (FILL THIS IN!) %%%%%
\newcommand{\authornames}{(Names)}

\newcommand{\coursenumber}{CSC 341}
\newcommand{\assignmentname}{Weekly Exercises \#13}
\newcommand{\assignmentversion}{1}

%%%%% Fonts and Encodings %%%%%%%%%%%%%%%%%%%%%%%%%%%%%%%%%%%%%%%%%%%%%%%%%%%%%%
\usepackage[T1]{fontenc}
\usepackage{libertine}

%%%%% General %%%%%%%%%%%%%%%%%%%%%%%%%%%%%%%%%%%%%%%%%%%%%%%%%%%%%%%%%%%%%%%%%%
\usepackage{amsmath}
\usepackage{amssymb}
\usepackage[shortlabels]{enumitem}
\setlist{nosep}
\usepackage{framed}
\usepackage{fancyvrb}
\usepackage[amsmath, amsthm, thmmarks]{ntheorem}
\usepackage{microtype}
\usepackage{url}
\usepackage[svgnames]{xcolor}
\usepackage{xspace}

\usepackage{tikz}

%%%%% Page Formatting %%%%%%%%%%%%%%%%%%%%%%%%%%%%%%%%%%%%%%%%%%%%%%%%%%%%%%%%%%
\usepackage[
  top=1in,
  bottom=1in,
  left=1in,
  right=1in,
  includefoot,
  paperwidth=8.5in,
  paperheight=11in
]{geometry}

\usepackage{fancyhdr}
\setlength{\headheight}{15.2pt}
\pagestyle{fancy}

\fancyhead{}
\fancyfoot{}
\lhead{\coursenumber{}---\assignmentname{} (ver. \assignmentversion), \authornames{}}
\cfoot{\thepage}

%%%%% Code %%%%%%%%%%%%%%%%%%%%%%%%%%%%%%%%%%%%%%%%%%%%%%%%%%%%%%%%%%%%%%%%%%%%%

\usepackage{listings}
\usepackage{xcolor}

% Adapted from: https://en.wikibooks.org/wiki/LaTeX/Source_Code_Listings
\definecolor{mygreen}{rgb}{0,0.6,0}
\definecolor{mygray}{rgb}{0.5,0.5,0.5}
\definecolor{mymauve}{rgb}{0.58,0,0.82}

\lstset{ 
  backgroundcolor=\color{white},      % choose the background color; you must add \usepackage{color} or \usepackage{xcolor}; should come as last argument
  basicstyle=\ttfamily\footnotesize,  % the size of the fonts that are used for the code
  breakatwhitespace=false,            % sets if automatic breaks should only happen at whitespace
  breaklines=true,                    % sets automatic line breaking
  captionpos=b,                       % sets the caption-position to bottom
  commentstyle=\color{mygreen},       % comment style
  keepspaces=true,                    % keeps spaces in text, useful for keeping indentation of code (possibly needs columns=flexible)
  keywordstyle=\color{blue},          % keyword style
  language=Python,                    % the language of the code
  stringstyle=\color{mymauve},        % string literal style
}

%%%%% Basic Macros and Definitions %%%%%%%%%%%%%%%%%%%%%%%%%%%%%%%%%%%%%%%%%%%%%
\newtheorem{claim}{Claim}
\newtheorem{invariant}{Invariant}
\newtheorem{defn}{Definition}
\newtheorem{thm}{Theorem}
\newtheorem{lemma}{Lemma}

\newcommand{\bigO}[1]{\ensuremath{\mathcal{O}(#1)}}

\newcommand{\ie}{\emph{i.e.}\xspace}
\newcommand{\eg}{\emph{e.g.}\xspace}
\newcommand{\hint}[1]{(\emph{Hint}: #1)}
\newcommand{\note}[1]{(\emph{Note}: #1)}

\newcounter{ProblemCounter}
\newenvironment{problem}[1][]
  {\refstepcounter{ProblemCounter}\noindent\textbf{Problem \theProblemCounter{} (#1)}\quad}
  {\newpage}

\newcommand{\answerbelow}{\noindent\makebox[\linewidth]{\rule{\textwidth}{0.4pt}}}

%%%%% Problem-specific Macros %%%%%%%%%%%%%%%%%%%%%%%%%%%%%%%%%%%%%%%%%%%%%%%%%%

\newcommand{\prob}[1]{\ensuremath{\textsf{#1}}\xspace}
\newcommand{\desc}[1]{\ensuremath{\langle #1 \rangle}}
\newcommand{\comp}[1]{\ensuremath{\overline{#1}}\xspace}

\newcommand{\nats}{\ensuremath{\mathbb{N}}\xspace}
\newcommand{\dfa}{\textsf{DFA}\xspace}
\newcommand{\EQdfa}{\ensuremath{EQ_\mathsf{DFA}\xspace}}
\newcommand{\Atm}{\ensuremath{A_\mathsf{TM}}\xspace}

%%%%%%%%%%%%%%%%%%%%%%%%%%%%%%%%%%%%%%%%%%%%%%%%%%%%%%%%%%%%%%%%%%%%%%%%%%%%%%%%

\begin{document}

\begin{problem}[Formation]

Before you begin the group activity for this week, identify:
\begin{itemize}
  \item A \emph{Scribe}: who will be responsible for creating a shared document
    and turning in your solutions for this week's problems.
  \item A \emph{Timekeeper}: who will keep track of the timer for weekly
    activities that require timekeeping.
  \item A \emph{Driver}: who will be responsible for asking people to share or
    otherwise kickstarting conversation if group chat slows down.
\end{itemize}

For this problem, simply identify your teammate's roles in the space below.

If you cannot attend a given class period, please work with your group on how
you will contribute to the group's work either asynchronously (\eg, by doing
the work yourself) or outside of the class period (\eg, by chatting or meeting
outside of class).  If there are any issues in coordinating work in your group,
please let me know as soon as possible!

\answerbelow
% FILL IN YOUR ANSWER HERE

\end{problem}

%%%%% Monday: Universality and Reversibility %%%%%%%%%%%%%%%%%%%%%%%%%%%%%%%%%%%

\begin{problem}[There And Back At Again]

In the reading, we introduced a circuit-based model of computation.  For these
exercises, we'll consider two important ideas related to this model:
\emph{universality} and \emph{reversibility}.

\begin{enumerate}
  \item As a warm-up, spend 5 minutes individually to design a circuit that
    computes the XOR of two input variables.  The XOR function returns 1 if
    exactly one of its inputs is a 1.  \hint{To do this quickly, express XOR in
    boolean logic in terms of \( (\wedge) \), \( (\vee) \), and \( (\neg) \)
    and then translate your resulting proposition into a circuit.}

    Come back as a group and compare your answers.  Give your final solution in
    your write-up.
  \item Example 9.29 of the book describes how this XOR circuit can be used to
    implement a family of parity circuits that detect whether an input of size
    \( n \) has an odd number of 1s.  With your group, spend 2--3 minutes
    discussing why this approach implies that the parity function has circuit
    complexity \bigO{n}.
  \item We say that a collection of gates is \emph{universal} if any circuit
    can be written in terms of these gates.  The book presents the circuit
    complexity model using AND, OR, and NOT gates.  These gates are universal
    for boolean logic, but it turns out we can do even better!  Consider the
    NAND gate.  It computes the negation of the AND operator as follows:
    \begin{center}
      \begin{tabular}{ccc}
        \textbf{X} & \textbf{Y} & \textbf{NAND} \\
        0          & 0          & 1             \\
        0          & 1          & 1             \\
        1          & 0          & 1             \\
        1          & 1          & 0
      \end{tabular}
    \end{center}
    Spent 10 minutes individually showing that the NAND gate alone is universal
    for boolean logic by showing how AND, OR, and NOT can be expressed in terms
    of NAND.  Feel free to assign one gate to each group member.  Come back as
    a group to share and agree upon your solutions, adding them to your
    write-up.
  \item We say that a logic gate \( f \) is \emph{reversible} if it is the case
    that for any output \( y \) there is a unique input \( x \) such that \(
    f(x) = y \).  As a consequence, there must exist an inverse gate \( f' \)
    such that \( f(y) = x \).  As a group, spend 10--15 minutes answering the
    following questions about reversible gates:
    \begin{itemize}
      \item Is OR a reversible gate?  Is XOR a reversible gate?  Why or why not
        in each case.
      \item Can any of our other binary operation gates be reversible, \eg,
        AND, OR, NAND, etc.?  Why or why not?
      \item With an answer to the previous question in mind, design a
        reversible version of the XOR gate.  The gate will have a different
        number of outputs than the traditional XOR gate but allow you to
        recover the XOR computation from the result.
    \end{itemize}
\end{enumerate}

\answerbelow
% FILL IN YOUR ANSWER HERE

\end{problem}

%%%%% Monday and Friday: Quantum Computation (Nothing to Do!) %%%%%%%%%%%%%%%%%%


\end{document}
