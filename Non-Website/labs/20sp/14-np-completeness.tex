\documentclass[11pt]{book}

\setcounter{chapter}{14}
\newcommand{\docclass}{CSC 341 (20sp)}
\newcommand{\doctitle}{NP-Completeness}
\newcommand{\docauthor}{Peter-Michael Osera}

\usepackage{reading}

\begin{document}

\begin{center}
  \large\textbf{{\doctitle}}
\end{center}

\vspace{2em}

%%%%% Problem-specific Macros %%%%%%%%%%%%%%%%%%%%%%%%%%%%%%%%%%%%%%%%%%%%%%%%%%

\newcommand{\desc}[1]{\ensuremath{\langle #1 \rangle}}
\newcommand{\NP}{\ensuremath{\mathsf{NP}}\xspace}
\newcommand{\lang}[1]{\ensuremath{\mathsf{#1}}\xspace}

%%%%%%%%%%%%%%%%%%%%%%%%%%%%%%%%%%%%%%%%%%%%%%%%%%%%%%%%%%%%%%%%%%%%%%%%%%%%%%%%

\begin{problem}{Bridging the Gap}
  Let \( A \) be a decidable language and \( V \) be a polynomial-time verifier for the language.
  Recall that a verifier is an algorithm that checks whether a candidate solution to the problem is indeed an actual solution.
  Give a generic non-deterministic turing machine \( N \) that decides \( A \) in polynomial time in terms of \( V \).
\end{problem}

\vspace{2.5in}

\begin{problem}{Membership}
  Show that each of the following languages are in \NP.
  \begin{enumerate}
    \item \( A_1 = \{ \desc{P} \mid \text{\( P \) is a satisfiable boolean formula} \} \).
    \item \( A_2 = \{ \desc{S, t} \mid \text{\( S \subseteq \mathbb{Z} \) and there exists \( x_1, \ldots, x_k \in S \) such that \( \sum_{i}^{k} x_i = t \)} \} \).
    \item \( A_3 = \{ \desc{G, V', k} \mid \text{\( G = (V, E) \) is a graph and \( V' \subseteq V \) is a \( k \)-covering of \( G \)} \} \).
  \end{enumerate}
\end{problem}

\newpage

\begin{problem}{Completeness}
  Define the following languages:
  \begin{align*}
    \lang{CLIQUE} =&\; \{ \desc{G, k} \mid \text{\( G \) is a graph and \( G \) contains a \( k \)-clique} \} \\
    \lang{ISO}    =&\; \{ \desc{G, H} \mid \text{\( G \) and \( H \) are graphs and there is a subset \( G' \) of \( G \) is isomorphic to \( H \)} \}.
  \end{align*}
  Show that \( \lang{CLIQUE} \leq_p \lang{ISO} \).
  (\emph{Hint}: what is the type of the mapping function \( f \) in this case?)
\end{problem}

\end{document}
