\documentclass[11pt]{book}

\setcounter{chapter}{15}
\newcommand{\docclass}{CSC 341 (20sp)}
\newcommand{\doctitle}{The Cook-Levine Theorem}
\newcommand{\docauthor}{Peter-Michael Osera}

\usepackage{reading}

\begin{document}

\begin{center}
  \large\textbf{{\doctitle}}
\end{center}

\vspace{2em}

%%%%% Problem-specific Macros %%%%%%%%%%%%%%%%%%%%%%%%%%%%%%%%%%%%%%%%%%%%%%%%%%

\newcommand{\desc}[1]{\ensuremath{\langle #1 \rangle}}
\newcommand{\NP}{\ensuremath{\mathsf{NP}}\xspace}
\newcommand{\lang}[1]{\ensuremath{\mathsf{#1}}\xspace}

%%%%%%%%%%%%%%%%%%%%%%%%%%%%%%%%%%%%%%%%%%%%%%%%%%%%%%%%%%%%%%%%%%%%%%%%%%%%%%%%

\begin{problem}{Details, Details, \ldots}
  Answer the following key questions about the proof of the Cook-Levine theorem that tests our understanding of Turing machines and polynominal-time reductions.
  \begin{enumerate}[label=(\alph*)]
    \item What are the inputs and outputs of the mapping function of the reduction?
    \item Why do the tableaus under consideration have \( n^k \) rows?
    \item Why can we bound the size of the tape, \ie, the number of rows of the tableau, to \( n^k \)?
    \item What do the boolean variables \( x_{i, j, s} \) represent in the reduction?
    \item What does each of the boolean formula represent: \( \phi_\mathsf{cell} \), \( \phi_{\mathsf{start}} \), \( \phi_{\mathsf{move}} \), \( \phi_{\mathsf{accept}} \)?
    \item Does a \( 2 \times 2 \) window work?
      Why or why not?
    \item How does the constructed boolean function account for nondeterminism?
  \end{enumerate}
\end{problem}

\end{document}
