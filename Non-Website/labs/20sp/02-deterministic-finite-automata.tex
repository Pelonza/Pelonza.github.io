\documentclass[11pt]{book}

\setcounter{chapter}{2}
\newcommand{\docclass}{CSC 341 (20sp)}
\newcommand{\doctitle}{Deterministic Finite Automata}
\newcommand{\docauthor}{Peter-Michael Osera}

\usepackage{reading}

\begin{document}

\begin{center}
  (Turn-in problems are indicated with a dagger \turninproblem{}.)
\end{center}

%%%%%%%%%%%%%%%%%%%%%%%%%%%%%%%%%%%%%%%%%%%%%%%%%%%%%%%%%%%%%%%%%%%%%%%%%%%%%%%%

\begin{problem}{Connect the Dots}

Answer the following questions about the fundamental definitions that connect languages, machines, and computation.

\vspace{1em}

\noindent (a) What do the metavariables \( Σ \), \( Σ^* \), and \( L \) traditionally represent?
  Give a concrete example that covers all three concepts.

\vspace{1in}

\noindent (b) What are the \emph{types} of the five components of a finite automata: \( (Q, \Sigma, \delta, q_0, F) \)?

\vspace{1in}

\noindent (c) What does \( L(M) \) conventionally mean?
  If \( L(M) = A \), what is the relationship between \( M \) and \( A \)?

\vspace{1in}

\noindent (d) If we want to \emph{prove} that ``\( M \) accepts \( w \)'', what we must we do?

\vspace{1in}

\end{problem}

\newpage

%%%%%%%%%%%%%%%%%%%%%%%%%%%%%%%%%%%%%%%%%%%%%%%%%%%%%%%%%%%%%%%%%%%%%%%%%%%%%%%%

\begin{problem}{Reasoning About DFAs}
  Consider the following DFA \( D \):
\begin{center}
  \begin{tikzpicture}[shorten >=1pt,node distance=4.0cm,on grid,auto]
    \node[state, initial]   (q_0)                {\( q_0 \)};
    \node[state]            (q_1) [right=of q_0] {\( q_1 \)};
    \node[state]            (q_2) [below=of q_0] {\( q_2 \)};
    \node[state, accepting] (q_3) [below=of q_1] {\( q_3 \)};
    \path[->]
      (q_0) edge [bend left] node {0} (q_3)
            edge [bend right] node {1} (q_2)
      (q_1) edge [bend right] node {0} (q_2)
            edge [bend left=45] node {1} (q_3)
      (q_2) edge [bend right] node {0} (q_1)
            edge [bend left=45] node {1} (q_0)
      (q_3) edge [bend left] node {0} (q_0)
            edge [bend right] node {1} (q_1);
  \end{tikzpicture}
\end{center}

\noindent (b) We can each think of state of a finite automata as encoding an \emph{property} about the current computation.
  Describe an appropriate property for each of \( q_0 \), \( q_1 \), \( q_2 \), and \( q_3 \).

\vspace{1in}

\noindent (a) Using these properties, give a formal description of \( L(D) \).

\vspace{1in}

\end{problem}

%%%%%%%%%%%%%%%%%%%%%%%%%%%%%%%%%%%%%%%%%%%%%%%%%%%%%%%%%%%%%%%%%%%%%%%%%%%%%%%%

\begin{problem}{closure}
  \turninproblem{} Define the \emph{reversal} of a string \( w = w_0 w_1 \cdots w_n \) to be \( w^R = w_n \cdots w_1 w_0 \).
  Using this definition, we can define the reversal of a language \( A \) to be \( L^R = \{ w^R \;|\; w \in L \} \).
Suppose that some DFA \( D \) recognizes \( L \), \ie, \( L(D) = A \).
Prove that there exists a DFA \( D' \) that recognizes \( L^R \).
\end{problem}

\end{document}
