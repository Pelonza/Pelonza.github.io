\documentclass[11pt]{book}

\setcounter{chapter}{8}
\newcommand{\docclass}{CSC 341 (20sp)}
\newcommand{\doctitle}{Context-free Grammars}
\newcommand{\docauthor}{Peter-Michael Osera}

\usepackage{reading}

\begin{document}

\begin{center}
  \large\textbf{{\doctitle}}
\end{center}

\vspace{2em}

%%%%%%%%%%%%%%%%%%%%%%%%%%%%%%%%%%%%%%%%%%%%%%%%%%%%%%%%%%%%%%%%%%%%%%%%%%%%%%%%

\begin{problem}{The Basics}

(Adapted from Sipser 2.3.)
Answer each part for the following context-free grammar G.

\begin{align*}
  R →&\; XRX \mid S \\
  S →&\; \texttt{a}T\texttt{b} \mid \texttt{b}T\texttt{a} \\
  T →&\; XTX \mid X \mid ϵ \\
  X →&\; \texttt{a} \mid \texttt{b}
\end{align*}

\begin{enumerate}[label=(\alph*)]
  \item What are the variables of \( G \)?
  \item What are the terminals of \( G \)?
  \item Which is the start variable of \( G \)?
  \item Give three strings in \( L(G) \).
  \item Give three strings \emph{not} in \( L(G) \).
  \item True or False: \( T \overset{*}{⇒} \texttt{aba} \).
    If true, give a derivation for the string.
  \item True of false: \( XXX \overset{*}{⇒} \texttt{aba} \).
    If true, give a derivation for the string.
  \item True or False: \( T \overset{*}{⇒} XX \).
  \item Give a description in English of \( L(G) \).
\end{enumerate}

\end{problem}

%%%%%%%%%%%%%%%%%%%%%%%%%%%%%%%%%%%%%%%%%%%%%%%%%%%%%%%%%%%%%%%%%%%%%%%%%%%%%%%%

\newpage

\begin{problem}{Conditional Ambiguity}

In C-like programming languages, conditional statements frequently have the following form:

\begin{align*}
  expr →&\; \texttt{true} \mid \texttt{false} \mid \cdots \\
  stmt →&\; \texttt{if}\;expr\;\texttt{then}\;stmt \mid \texttt{if}\;expr\;\texttt{then}\;stmt\;\texttt{else}\;stmt \mid \texttt{skip} \mid \cdots
\end{align*}

(The \texttt{skip} statement is a statement that does nothing.
For our purposes, it acts as a "base case" to the \( stmt \) production.
You could substitute any "real" base statement such as a call to a function, \eg, \texttt{printf} instead.)

\begin{enumerate}
  \item Give a string and two parse trees demonstrating the ambiguity of this grammar.
  \item Which of the two parse trees is the "preferred" interpretation?
  \item Translate your toy example to an equivalent C program and try it out, \eg, with \texttt{gcc} or \texttt{clang}.
    Which of the two parse trees is the one that your chosen compiler produces?
  \item Change the grammar above to remove the ambiguity.
    You may try to preserve the original syntax of the conditional statement, but you may also add additional syntax to remove the ambiguity.
    Your solution should be as faithful to good C programming practices as possible.
\end{enumerate}

\end{problem}

\end{document}
