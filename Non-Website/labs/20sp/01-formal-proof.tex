\documentclass[11pt]{book}

\setcounter{chapter}{0}
\newcommand{\docclass}{CSC 341 (20sp)}
\newcommand{\doctitle}{Daily Exercise---Formal Proof}
\newcommand{\docauthor}{Peter-Michael Osera}

\usepackage{reading}

\begin{document}

\chapter{\doctitle}

\section{About Course Daily Work}

Our class periods will involve daily work in class to encourage you to tackle the material systematically.
We'll approach this work in a variety of ways, \eg, individual work, class activities, lead-by-example, but most commonly, we'll work in randomly-assigned pairs.
While this is a mathematics-oriented course, group work is essential to help you develop your understanding of the material.
When approaching group work, try not to work individually and then compare answers at the end!
Instead, consider the following strategies for making collaboration on these problems more beneficial to you and your partner:
\begin{itemize}
  \item Vocalize your thought process to your partner.
    Try to explain what you are thinking about using words or pictures.
  \item If something your partner says confuses you, or you don't understand what's going on, ask a question!
    Remember that good mathematics is elegant and simple; you help your partner by forcing them to try to explain things in such a manner.
  \item If you feel like your partner is slowing down your progress, keep in mind that (a) daily work is not a race! and (b) trying to help a peer get through the material helps clarify and reinforce the material for yourself as well!
\end{itemize}

\subsection{Turning In Daily Work}

Each set of daily exercises contains \emph{write-up problems} indicated with a dagger \turninproblem{} that I will ask you to turn in to Gradescope.
You should polish your answers to these problems outside of class, write them up in \LaTeX{} and turn them into Gradescope online.
Your write-ups to these problems along with your solutions to the reading exercises for the next class period will generally be due before the next period.

In class, I encourage you to write your initial work by-hand on the worksheets I provide.
While you will need to turn in the required daily work problems through a \LaTeX{}-generated PDF, I find that trying to typeset solutions distracts me from deriving said solutions.
Furthermore, you will find it immensely helpful to draw graphs and pictures to organize your thoughts, something that is difficult to do in \LaTeX{}.

To turn in your daily work, please use the daily work \LaTeX{} template found on the course website.
Make sure to fill in the macros at the top of the file to add your name and the title of the assignment to your write-up.
The template also has macros for giving your solutions to each problem as well as citing your collaborators and resources.

\newpage

\section{Formal Proof}

Formally prove the following claims over strings.

\begin{claim}
  Call an element \( u \) of a set \( S \) and an associated binary operation over that set \( (+) : S → S \) a \emph{unit} if \( ∀x ∈ S.\, x + u = u + x = x \).
  \( ϵ \) is a unit for strings and concatenation over strings.
\end{claim}

\begin{claim}
  Let \( x \) and \( y \) be strings.
  If \( x \) is a prefix of \( y \) and \( y \) is a prefix of \( x \) then \( x = y \).
\end{claim}

\newpage

\section{Induction Revisited}

Consider the following definitions regarding binary trees.
\begin{defn}[Binary Tree]
  A binary tree is either:
  \begin{itemize}
    \item A \emph{leaf} or
    \item A \emph{node} with two sub-trees, its \emph{children}.
  \end{itemize}
  We define the \emph{root} of a binary tree to be a node that is not the child of any other node in the tree.
\end{defn}
\begin{defn}[Levels and Heights]
  The \emph{level} of a node in a binary tree is the length of the path from the root to that node.
  The \emph{height} of a binary tree is the maximal level of any node in the tree.
  We also use ``level'' to denote the \emph{set of all nodes} that share the same level in the tree.
\end{defn}
\begin{defn}[Complete, Perfect Trees]
  A tree is \emph{complete} if each level of the tree contains its maximal number of possible nodes.
  A tree is \emph{perfect} if each node of the tree contains zero or two children.
\end{defn}

Now prove the following claim by structural induction.
\begin{claim}
  Let \( h \) be the height of a complete, perfect binary tree.
  If \( n ≤ h \), then there are \( 2^n \) nodes at level \( n \) of this tree.
\end{claim}

\newpage

\section{Constructive Proof}

\turninproblem{}
Chess is played on a \( n × n \) board (usually \( n = 8 \)).
Consider the \emph{rook} piece, which can move in any number of squares in a cardinal (non-diagonal) direction.
A \emph{rook's tour} is a sequence of moves for a single rook that causes the piece to visit \emph{every} square of the board \emph{exactly once}.
When considering a tour, we are free to place our piece on the board at any position initially.
Also, we only consider a square visited if the piece \emph{ends its movement} on that square.

Prove via induction that there exists a rook's tour for any chessboard of size \( n ≥ 1 \).
(\emph{Hint:} pay attention to the details!
This is not as straightforward of a proof as you might think!)

\end{document}
