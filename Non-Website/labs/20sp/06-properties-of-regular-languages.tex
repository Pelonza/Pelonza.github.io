\documentclass[11pt]{book}

\setcounter{chapter}{6}
\newcommand{\docclass}{CSC 341 (20sp)}
\newcommand{\doctitle}{Properties of the Regular Models}
\newcommand{\docauthor}{Peter-Michael Osera}

\usepackage{reading}

\begin{document}

\begin{center}
  (Turn-in problems are indicated with a dagger \turninproblem{}.)
\end{center}

\begin{center}
  \large\textbf{{\doctitle}}
\end{center}

\vspace{2em}

%%%%%%%%%%%%%%%%%%%%%%%%%%%%%%%%%%%%%%%%%%%%%%%%%%%%%%%%%%%%%%%%%%%%%%%%%%%%%%%%

\newcommand{\vect}[3]{\begin{bmatrix}#1 \\ #2 \\ #3\end{bmatrix}}

\begin{problem}{All-NFAs}

(Sipser 1.38)
An \emph{all}-NFA \( M \) is a 5-tuple \( (Q, \Sigma, \delta, q_0, F) \) that accepts \( x \in \Sigma^* \) if \emph{every} possible state that \( M \) could be in after reading input \( x \) is a state from \( F \).
Note, in contrast, that an ordinary NFA accepts a string if \emph{some} state among these possible states is an accept state.
Prove that all-NFAs recognize the class of regular languages.
(\emph{Hint}: you need to prove equivalence in \emph{both} directions.)

\vspace{2in}

\begin{problem}{That's Regular?}

(Sipser 1.32)
Let
\[
  \Sigma_3 = \Bigg\{ \vect{0}{0}{0}, \vect{0}{0}{1}, \vect{0}{1}{0}, \ldots, \vect{1}{1}{1} \Bigg\}.
\]
\( \Sigma_3 \) contains all size 3 columsn of 0s and 1s.
A string of symbols in \( \Sigma_3 \) gives three rows of 0s and 1s.
Consider each row to be a binary number and let
\[
  B = \{ w \in \Sigma^*_3 \mid \text{The bottom row of \( w \) is the sum of the top two rows} \}.
\]
For example,
\[
  \vect{0}{0}{1} \vect{1}{0}{0} \vect{1}{1}{0} \in B
\]
but
\[
  \vect{0}{0}{1} \vect{1}{0}{1} \not\in B.
\]
Show that \( B \) is regular.
To do this, note that we've already proven language reversal, \( L^R \), is closed under the regular languages.
You will likely find it easier to work with \( B^R \) instead.

\end{problem}

\newpage

\end{problem}

\begin{problem}{All Or Nothing \turninproblem}

Give an algorithm to determine whether a DFA \( D \) accepts \emph{all} strings, \ie, \( L(D) = Σ^* \).
Argue informally why your algorithm works.

\end{problem}

\end{document}
