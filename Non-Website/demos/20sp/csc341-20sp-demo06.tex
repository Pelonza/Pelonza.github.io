\documentclass[12pt]{article}

%%%%% Document Information (FILL THIS IN!) %%%%%
\newcommand{\authorname}{(Name)}
\newcommand{\grinnellusername}{(Username)}

\newcommand{\coursenumber}{CSC 341}
\newcommand{\assignmentname}{Demo Exercises \#6}
\newcommand{\assignmentversion}{1}

%%%%% Fonts and Encodings %%%%%%%%%%%%%%%%%%%%%%%%%%%%%%%%%%%%%%%%%%%%%%%%%%%%%%
\usepackage[T1]{fontenc}
\usepackage{libertine}

%%%%% General %%%%%%%%%%%%%%%%%%%%%%%%%%%%%%%%%%%%%%%%%%%%%%%%%%%%%%%%%%%%%%%%%%
\usepackage{amsmath}
\usepackage{amssymb}
\usepackage{enumitem}
\usepackage{fancyvrb}
\usepackage[amsmath, amsthm, thmmarks]{ntheorem}
\usepackage{microtype}
\usepackage{url}
\usepackage[svgnames]{xcolor}
\usepackage{xspace}

\usepackage{tikz}

%%%%% Page Formatting %%%%%%%%%%%%%%%%%%%%%%%%%%%%%%%%%%%%%%%%%%%%%%%%%%%%%%%%%%
\usepackage[
  top=1in,
  bottom=1in,
  left=1in,
  right=1in,
  includefoot,
  paperwidth=8.5in,
  paperheight=11in
]{geometry}

\usepackage{fancyhdr}
\setlength{\headheight}{15.2pt}
\pagestyle{fancy}

\fancyhead{}
\fancyfoot{}
\lhead{\coursenumber{}---\assignmentname{} (ver. \assignmentversion), \authorname{} [\grinnellusername{}]}
\cfoot{\thepage}

%%%%% Code %%%%%%%%%%%%%%%%%%%%%%%%%%%%%%%%%%%%%%%%%%%%%%%%%%%%%%%%%%%%%%%%%%%%%

\usepackage{listings}
\usepackage{xcolor}

% Adapted from: https://en.wikibooks.org/wiki/LaTeX/Source_Code_Listings
\definecolor{mygreen}{rgb}{0,0.6,0}
\definecolor{mygray}{rgb}{0.5,0.5,0.5}
\definecolor{mymauve}{rgb}{0.58,0,0.82}

\lstset{ 
  backgroundcolor=\color{white},      % choose the background color; you must add \usepackage{color} or \usepackage{xcolor}; should come as last argument
  basicstyle=\ttfamily\footnotesize,  % the size of the fonts that are used for the code
  breakatwhitespace=false,            % sets if automatic breaks should only happen at whitespace
  breaklines=true,                    % sets automatic line breaking
  captionpos=b,                       % sets the caption-position to bottom
  commentstyle=\color{mygreen},       % comment style
  keepspaces=true,                    % keeps spaces in text, useful for keeping indentation of code (possibly needs columns=flexible)
  keywordstyle=\color{blue},          % keyword style
  language=Java,                      % the language of the code
  stringstyle=\color{mymauve},        % string literal style
}

%%%%% Basic Macros and Definitions %%%%%%%%%%%%%%%%%%%%%%%%%%%%%%%%%%%%%%%%%%%%%
\newtheorem{claim}{Claim}
\newtheorem{invariant}{Invariant}
\newtheorem{defn}{Definition}
\newtheorem{thm}{Theorem}
\newtheorem{lemma}{Lemma}

\newcommand{\ie}{\emph{i.e.}\xspace}
\newcommand{\eg}{\emph{e.g.}\xspace}
\newcommand{\etc}{\emph{etc.}\xspace}
\newcommand{\hint}[1]{(\emph{Hint}: #1)}

\newcounter{ProblemCounter}
\newenvironment{problem}[1][]
  {\refstepcounter{ProblemCounter}\noindent\textbf{Problem \theProblemCounter{} (#1)}\quad}
  {\newpage}

\newcommand{\answerbelow}{\noindent\makebox[\linewidth]{\rule{\textwidth}{0.4pt}}}


%%%%% Problem-specific Macros %%%%%%%%%%%%%%%%%%%%%%%%%%%%%%%%%%%%%%%%%%%%%%%%%%

\newcommand{\np}{\ensuremath{\mathsf{NP}}\xspace}
\newcommand{\prob}[1]{\ensuremath{\mathsf{#1}}\xspace}
\newcommand{\desc}[1]{\ensuremath{\langle #1 \rangle}}
\newcommand{\Nat}{\ensuremath{\mathbb{N}}\xspace}
\newcommand{\comp}[1]{\ensuremath{\overline{#1}}\xspace}

%%%%%%%%%%%%%%%%%%%%%%%%%%%%%%%%%%%%%%%%%%%%%%%%%%%%%%%%%%%%%%%%%%%%%%%%%%%%%%%%

\begin{document}

\begin{problem}[Two-Player Games]

Complete problems 8.2 and 8.3 from Sipser that discuss strategies for
two-player games (tic-tac-toe and generalized geography).

\answerbelow{}

% FILL IN YOUR ANSWER HERE

\end{problem}

%%%%%%%%%%%%%%%%%%%%%%%%%%%%%%%%%%%%%%%%%%%%%%%%%%%%%%%%%%%%%%%%%%%%%%%%%%%%%%%%

\begin{problem}[Reductions]

Prove that the following languages are undecidable by reduction from a known,
undecidable language:
\begin{enumerate}[itemsep=0pt, label=(\alph*)]
  \item \( L_1 = \{ \desc{M} \mid \text{\( M \) is a TM and \( L(M) \) is infinite} \} \)
      \item \( L_2 = \{ \desc{M} \mid \text{\( M \) is a TM and the \( L(M) \) is finite} \} \)
\end{enumerate}

\hint{A contradiction can be established with a simple mapping reduction for \(
L_1 \).  However, \( L_2 \) is surprisingly not as simple; you will need to
write a general reduction instead!}

\answerbelow

% FILL IN YOUR ANSWER HERE

\end{problem}

%%%%%%%%%%%%%%%%%%%%%%%%%%%%%%%%%%%%%%%%%%%%%%%%%%%%%%%%%%%%%%%%%%%%%%%%%%%%%%%%

\begin{problem}[Machinations]

Consider two problems:
\begin{itemize}[itemsep=0pt]
  \item Determine if a TM \( M \) on input \( w \) ever attempts to move its
    head left when its head is on the left-most tape cell.
  \item Determine if a TM \( M \) on
    input \( w \) ever attempts to move its head left during computation on \(
    w \).
\end{itemize}
\begin{enumerate}[itemsep=0pt, label=(\alph*)]
  \item Formulate these two problems as languages \( L_1 \) and \( L_2 \).
  \item \( L_2 \) is decidable; prove it.
  \item \( L_1 \) is undecidable; prove it.
  \item With what you learned from these proofs, come up with a language \( L_3
    \) that is also a deciable problem that analyzes how a TM \( M \) computes
    on input \( w \).  That is, the problem should not concern the language of
    \( M \) but instead, for example, what \( M \) writes to the tape, how \( M
    \) moves its tape head, \etc
\end{enumerate}

\answerbelow

% FILL IN YOUR ANSWER HERE

\end{problem}

%%%%%%%%%%%%%%%%%%%%%%%%%%%%%%%%%%%%%%%%%%%%%%%%%%%%%%%%%%%%%%%%%%%%%%%%%%%%%%%%

\end{document}
