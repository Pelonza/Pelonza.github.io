\documentclass[12pt]{article}

%%%%% Document Information (FILL THIS IN!) %%%%%
\newcommand{\authorname}{(Name)}
\newcommand{\grinnellusername}{(Username)}

\newcommand{\coursenumber}{CSC 341}
\newcommand{\assignmentname}{Demo Exercises \#1}
\newcommand{\assignmentversion}{1}

%%%%% Fonts and Encodings %%%%%%%%%%%%%%%%%%%%%%%%%%%%%%%%%%%%%%%%%%%%%%%%%%%%%%
\usepackage[T1]{fontenc}
\usepackage{libertine}

%%%%% General %%%%%%%%%%%%%%%%%%%%%%%%%%%%%%%%%%%%%%%%%%%%%%%%%%%%%%%%%%%%%%%%%%
\usepackage{amsmath}
\usepackage{amssymb}
\usepackage{enumitem}
\usepackage{fancyvrb}
\usepackage[amsmath, amsthm, thmmarks]{ntheorem}
\usepackage{microtype}
\usepackage{url}
\usepackage[svgnames]{xcolor}
\usepackage{xspace}

%%%%% Page Formatting %%%%%%%%%%%%%%%%%%%%%%%%%%%%%%%%%%%%%%%%%%%%%%%%%%%%%%%%%%
\usepackage[
  top=1in,
  bottom=1in,
  left=1in,
  right=1in,
  includefoot,
  paperwidth=8.5in,
  paperheight=11in
]{geometry}

\usepackage{fancyhdr}
\setlength{\headheight}{15.2pt}
\pagestyle{fancy}

\fancyhead{}
\fancyfoot{}
\lhead{\coursenumber{}---\assignmentname{} (ver. \assignmentversion), \authorname{} [\grinnellusername{}]}
\cfoot{\thepage}

%%%%% Code %%%%%%%%%%%%%%%%%%%%%%%%%%%%%%%%%%%%%%%%%%%%%%%%%%%%%%%%%%%%%%%%%%%%%

\usepackage{listings}
\usepackage{xcolor}

% Adapted from: https://en.wikibooks.org/wiki/LaTeX/Source_Code_Listings
\definecolor{mygreen}{rgb}{0,0.6,0}
\definecolor{mygray}{rgb}{0.5,0.5,0.5}
\definecolor{mymauve}{rgb}{0.58,0,0.82}

\lstset{ 
  backgroundcolor=\color{white},      % choose the background color; you must add \usepackage{color} or \usepackage{xcolor}; should come as last argument
  basicstyle=\ttfamily\footnotesize,  % the size of the fonts that are used for the code
  breakatwhitespace=false,            % sets if automatic breaks should only happen at whitespace
  breaklines=true,                    % sets automatic line breaking
  captionpos=b,                       % sets the caption-position to bottom
  commentstyle=\color{mygreen},       % comment style
  keepspaces=true,                    % keeps spaces in text, useful for keeping indentation of code (possibly needs columns=flexible)
  keywordstyle=\color{blue},          % keyword style
  language=Java,                      % the language of the code
  stringstyle=\color{mymauve},        % string literal style
}

%%%%% Basic Macros and Definitions %%%%%%%%%%%%%%%%%%%%%%%%%%%%%%%%%%%%%%%%%%%%%
\newtheorem{claim}{Claim}
\newtheorem{invariant}{Invariant}
\newtheorem{defn}{Definition}
\newtheorem{thm}{Theorem}

\newcommand{\ie}{\emph{i.e.}\xspace}
\newcommand{\eg}{\emph{e.g.}\xspace}

\newcounter{ProblemCounter}
\newenvironment{problem}[1][]
  {\refstepcounter{ProblemCounter}\noindent\textbf{Problem \theProblemCounter{} (#1)}\quad}
  {\newpage}

\newcommand{\answerbelow}{\noindent\makebox[\linewidth]{\rule{\textwidth}{0.4pt}}}

\begin{document}

%%%%%%%%%%%%%%%%%%%%%%%%%%%%%%%%%%%%%%%%%%%%%%%%%%%%%%%%%%%%%%%%%%%%%%%%%%%%%%%%

\begin{problem}[Protocols]

A powerful application of finite automata is the specification and
verification of \emph{protocols}.  In this problem, we'll examine a protocol
between a client and a conference management server (CMS) such as
HotCRP\footnote{%
  \url{https://hotcrp.com}
}.
A client to this system may be an \emph{author} editing a paper for a review
That same client might also be a \emph{reviewer} editing reviews for papers
submitted to the conference.  Such a system segregates the two \emph{roles} so
that authors cannot edit reviews of other papers they are in competition with.

The client may issue the following commands to the server:
\begin{itemize}[itemsep=0pt]
  \item Connect
  \item Change role to author
  \item Change role to reviewer
  \item Edit paper
  \item Edit review
  \item Disconnect
\end{itemize}
A valid \emph{session} between the client and CMS adheres to the following
rules.
\begin{enumerate}[itemsep=0pt]
  \item The client must connect to the server before performing any other
    action.
  \item Once connected, the client is initially given the author role.
  \item While connected, the client can switch their role between author
    and reviewer.  (Presumably, the client needs to have proper rights to
    switch roles, but that is not captured in this protocol.)
  \item As an author, the client can edit papers but not reviews
  \item As a reviewer, the client can edit reviews but not papers.
  \item The client cannot connect to the server if it is already connected.
  \item The client must disconnect from the server to end the session.
\end{enumerate}
Note that in a single session, the client may connect and disconnect multiple
times.  However, every valid session ends with the client disconnected from the
CMS.

\begin{enumerate}[itemsep=0pt, label=(\alph*)]
  \item Specify a language \( L \) that captures valid sessions between the
    client and the server.  \( L \) should be specified in set-theoretic terms
    without appeal to a machine, and you should clearly define its alphabet
    \( \Sigma \).
  \item Give a deterministic finite automata \( D \) that recognizes \( L \).
    You may use the TikZ library to render the DFA with LaTeX or the
    \texttt{graphicx} package to include the DFA as an image.  You may consult
    online resources in using these packages.
  \item Prove that the DFA \( D \) recognizes \( L \).  To do this, assign
    relevant properties to each state of \( D \) and perform an exhaustive case
    analysis on each state \( q \) of \( D \), arguing that each of \( q \)'s
    transitions is valid and preserves the property of \( q \) according to the
    rules of the protocol.
  \item Give an implementation of the protocol in a real-world programming
    language of your choice, inlining the code below using the
    \texttt{verbatim} environment (or the \texttt{listings} package if you
    don't mind looking up more documentation).  Realize the protocol as a
    function that takes as input:
    \begin{itemize}
      \item The current state of the client and
      \item The next command issued by the client
    \end{itemize}
    and produces the next state of the client as output.  You may choose any
    representation of states and commands appropriate for your chosen language.
  \item Finally, based on your experiences developing \( D \) and your
    ``real-world'' implementation of the protocol, describe the pros and cons
    of first using a finite automata to \emph{model} a protocol \emph{before}
    going to implementation.  Be open-minded about the benefits you consider
    not just in terms of correctness but also productivity.
\end{enumerate}

\answerbelow{}

%%%%% FILL IN YOUR ANSWERS HERE %%%%%

\end{problem}

\end{document}
