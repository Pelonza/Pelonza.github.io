\documentclass[12pt]{article}

%%%%% Document Information (FILL THIS IN!) %%%%%
\newcommand{\authorname}{(Name)}
\newcommand{\grinnellusername}{(Username)}

\newcommand{\coursenumber}{CSC 341}
\newcommand{\assignmentname}{Demo Exercises \#3}
\newcommand{\assignmentversion}{1}

%%%%% Fonts and Encodings %%%%%%%%%%%%%%%%%%%%%%%%%%%%%%%%%%%%%%%%%%%%%%%%%%%%%%
\usepackage[T1]{fontenc}
\usepackage{libertine}

%%%%% General %%%%%%%%%%%%%%%%%%%%%%%%%%%%%%%%%%%%%%%%%%%%%%%%%%%%%%%%%%%%%%%%%%
\usepackage{amsmath}
\usepackage{amssymb}
\usepackage{enumitem}
\usepackage{fancyvrb}
\usepackage[amsmath, amsthm, thmmarks]{ntheorem}
\usepackage{microtype}
\usepackage{url}
\usepackage[svgnames]{xcolor}
\usepackage{xspace}

%%%%% Page Formatting %%%%%%%%%%%%%%%%%%%%%%%%%%%%%%%%%%%%%%%%%%%%%%%%%%%%%%%%%%
\usepackage[
  top=1in,
  bottom=1in,
  left=1in,
  right=1in,
  includefoot,
  paperwidth=8.5in,
  paperheight=11in
]{geometry}

\usepackage{fancyhdr}
\setlength{\headheight}{15.2pt}
\pagestyle{fancy}

\fancyhead{}
\fancyfoot{}
\lhead{\coursenumber{}---\assignmentname{} (ver. \assignmentversion), \authorname{} [\grinnellusername{}]}
\cfoot{\thepage}

%%%%% Code %%%%%%%%%%%%%%%%%%%%%%%%%%%%%%%%%%%%%%%%%%%%%%%%%%%%%%%%%%%%%%%%%%%%%

\usepackage{listings}
\usepackage{xcolor}

% Adapted from: https://en.wikibooks.org/wiki/LaTeX/Source_Code_Listings
\definecolor{mygreen}{rgb}{0,0.6,0}
\definecolor{mygray}{rgb}{0.5,0.5,0.5}
\definecolor{mymauve}{rgb}{0.58,0,0.82}

\lstset{ 
  backgroundcolor=\color{white},      % choose the background color; you must add \usepackage{color} or \usepackage{xcolor}; should come as last argument
  basicstyle=\ttfamily\footnotesize,  % the size of the fonts that are used for the code
  breakatwhitespace=false,            % sets if automatic breaks should only happen at whitespace
  breaklines=true,                    % sets automatic line breaking
  captionpos=b,                       % sets the caption-position to bottom
  commentstyle=\color{mygreen},       % comment style
  keepspaces=true,                    % keeps spaces in text, useful for keeping indentation of code (possibly needs columns=flexible)
  keywordstyle=\color{blue},          % keyword style
  language=Java,                      % the language of the code
  stringstyle=\color{mymauve},        % string literal style
}

%%%%% Basic Macros and Definitions %%%%%%%%%%%%%%%%%%%%%%%%%%%%%%%%%%%%%%%%%%%%%
\newtheorem{claim}{Claim}
\newtheorem{invariant}{Invariant}
\newtheorem{defn}{Definition}
\newtheorem{thm}{Theorem}
\newtheorem{lemma}{Lemma}

\newcommand{\ie}{\emph{i.e.}\xspace}
\newcommand{\eg}{\emph{e.g.}\xspace}

\newcounter{ProblemCounter}
\newenvironment{problem}[1][]
  {\refstepcounter{ProblemCounter}\noindent\textbf{Problem \theProblemCounter{} (#1)}\quad}
  {\newpage}

\newcommand{\answerbelow}{\noindent\makebox[\linewidth]{\rule{\textwidth}{0.4pt}}}


%%%%% Problem-specific Macros %%%%%%%%%%%%%%%%%%%%%%%%%%%%%%%%%%%%%%%%%%%%%%%%%%

\DeclareMathOperator{\Embeds}{\textsf{embeds}}

%%%%%%%%%%%%%%%%%%%%%%%%%%%%%%%%%%%%%%%%%%%%%%%%%%%%%%%%%%%%%%%%%%%%%%%%%%%%%%%%

\begin{document}

%%%%%%%%%%%%%%%%%%%%%%%%%%%%%%%%%%%%%%%%%%%%%%%%%%%%%%%%%%%%%%%%%%%%%%%%%%%%%%%%

\begin{problem}[Outside the Box]

The pumping lemma for context-free language extends the notion of ``pumping'' a
single pattern in a regular language to mutiple patterns in a context-free
language.

\begin{lemma}[Pumping Lemma (Context-free) Languages]
  If \( A \) is context-free, then there exists a pumping length \( p \in
  \mathbb{N} \) such that for any string \( s \in A \) with \( |s| \geq p \),
  it may be decomposed into \( s = uvxyz \) with the following constraints:
  \begin{enumerate}
    \item \( \forall i \geq 0 \), \( uv^{i}xy^{i}z \in A \).
    \item \( |vy| > 0 \).
    \item \( |vxy| \leq p \).
  \end{enumerate}
\end{lemma}
Consider the language \( L = \{ a^i b^j c^k \mid i < j \wedge i < k \} \) and the
following partial proof that \( L \) is not context-free with the pumping
lemma:
\begin{proof}
  Assume for the sake of contradiction that \( L \) is context-free.  The pumping
  lemma applies so there exists a pumping length \( p \) for which the lemma
  applies.  Consider \( s = a^{p} b^{p+1} c^{p+1} \in s, |s| \geq p \).
\end{proof}

\begin{enumerate}[label=(\alph*)]
  \item We can think of the constraints of the pumping stating that (i) at
    least one of \( v \) and \( y \) is non-empty and (ii) the substring \( vxy
    \) is an (at-most) \( p \) sized substring of \( s \).  With this in mind,
    describe all the possible the shapes of \( vxy \).  (\emph{Hint}: think of
    \( vxy \) as a sliding window across \( s \) \ldots)
  \item Consider one of these cases: \( vxy \) rests entirely in the region of
    \( a \)s. Apply the first constraint of the pumping lemma to arrive at a
    contradiction in this case of the proof.
  \item Complete the remaining cases of the proof, demonstrating that no matter
    what case we're in, we arrive at a contradiction. (\emph{Hint}: you don't
    have to make a similar argument for every case that you listed above.
    Think about how the cases are related and group them together to make your
    proof more concise.)
\end{enumerate}

% FILL IN YOUR ANSWER HERE

\end{problem}

%%%%%%%%%%%%%%%%%%%%%%%%%%%%%%%%%%%%%%%%%%%%%%%%%%%%%%%%%%%%%%%%%%%%%%%%%%%%%%%%

\begin{problem}[Kata]
  Turing machines are theoretically capable of any computation that a
  traditional computing device can perform.  However it is sometimes difficult
  to see that when a Turing machine is seemingly so minimalistic.  In this
  problem, we'll explore how to implement common patterns of behavior with a
  Turing machine that serve as building blocks for more complicated behaviors.

  For each of these descriptions of Turing machine behavior, give an
  implementation-level description of a Turing machine as a diagram that
  immediately exhibits that behavior.  The Turing machine does not need to
  accept any particular language beyond what is explicitly specified in the
  description.  In all cases, assume \( \Sigma = \{ 0, 1 \} \) and allow the
  tape alphabet to expand to handle any additional characters required by the
  descriptions.
  \begin{enumerate}[itemsep=0pt, label=(\alph*)]
    \item Traverse the tape and place a dollar sign (\$) at the \emph{end} of the tape.
    \item Sweep the entire tape, replace each $0$ with an $x$.
    \item Place a dollar sign (\$) at the \emph{beginning} of the tape.
    \item Insert a blank symbol in between every character of the tape.  If the
      tape has shape:
      \[
        w_0 \cdots w_k
      \]
      Then the tape contents should be:
      \[
        w_0\texttt{\textvisiblespace}w_1\texttt{\textvisiblespace}\cdots\texttt{\textvisiblespace}w_k
      \]
      where ``\texttt{\textvisiblespace}'' represents a blank symbol.
    \item Duplicate the tape, separating the copy from the original with a
      dollar sign (\$).  If the input string has shape:
      \[
        w_0 \cdots w_k
      \]
      Then the tape contents should be:
      \[
        w_0 \cdots w_k \$ w_0 \cdots w_k
      \]
      after you are done.
  \end{enumerate}

  % FILL IN YOUR ANSWER HERE

\end{problem}

%%%%%%%%%%%%%%%%%%%%%%%%%%%%%%%%%%%%%%%%%%%%%%%%%%%%%%%%%%%%%%%%%%%%%%%%%%%%%%%%

\end{document}
