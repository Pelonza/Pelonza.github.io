\documentclass[12pt]{article}

%%%%% Document Information (FILL THIS IN!) %%%%%
\newcommand{\authorname}{(Name)}
\newcommand{\grinnellusername}{(Username)}

\newcommand{\coursenumber}{CSC 341}
\newcommand{\assignmentname}{Demo Exercises \#4}
\newcommand{\assignmentversion}{1}

%%%%% Fonts and Encodings %%%%%%%%%%%%%%%%%%%%%%%%%%%%%%%%%%%%%%%%%%%%%%%%%%%%%%
\usepackage[T1]{fontenc}
\usepackage{libertine}

%%%%% General %%%%%%%%%%%%%%%%%%%%%%%%%%%%%%%%%%%%%%%%%%%%%%%%%%%%%%%%%%%%%%%%%%
\usepackage{amsmath}
\usepackage{amssymb}
\usepackage{enumitem}
\usepackage{fancyvrb}
\usepackage[amsmath, amsthm, thmmarks]{ntheorem}
\usepackage{microtype}
\usepackage{url}
\usepackage[svgnames]{xcolor}
\usepackage{xspace}

%%%%% Page Formatting %%%%%%%%%%%%%%%%%%%%%%%%%%%%%%%%%%%%%%%%%%%%%%%%%%%%%%%%%%
\usepackage[
  top=1in,
  bottom=1in,
  left=1in,
  right=1in,
  includefoot,
  paperwidth=8.5in,
  paperheight=11in
]{geometry}

\usepackage{fancyhdr}
\setlength{\headheight}{15.2pt}
\pagestyle{fancy}

\fancyhead{}
\fancyfoot{}
\lhead{\coursenumber{}---\assignmentname{} (ver. \assignmentversion), \authorname{} [\grinnellusername{}]}
\cfoot{\thepage}

%%%%% Code %%%%%%%%%%%%%%%%%%%%%%%%%%%%%%%%%%%%%%%%%%%%%%%%%%%%%%%%%%%%%%%%%%%%%

\usepackage{listings}
\usepackage{xcolor}

% Adapted from: https://en.wikibooks.org/wiki/LaTeX/Source_Code_Listings
\definecolor{mygreen}{rgb}{0,0.6,0}
\definecolor{mygray}{rgb}{0.5,0.5,0.5}
\definecolor{mymauve}{rgb}{0.58,0,0.82}

\lstset{ 
  backgroundcolor=\color{white},      % choose the background color; you must add \usepackage{color} or \usepackage{xcolor}; should come as last argument
  basicstyle=\ttfamily\footnotesize,  % the size of the fonts that are used for the code
  breakatwhitespace=false,            % sets if automatic breaks should only happen at whitespace
  breaklines=true,                    % sets automatic line breaking
  captionpos=b,                       % sets the caption-position to bottom
  commentstyle=\color{mygreen},       % comment style
  keepspaces=true,                    % keeps spaces in text, useful for keeping indentation of code (possibly needs columns=flexible)
  keywordstyle=\color{blue},          % keyword style
  language=Java,                      % the language of the code
  stringstyle=\color{mymauve},        % string literal style
}

%%%%% Basic Macros and Definitions %%%%%%%%%%%%%%%%%%%%%%%%%%%%%%%%%%%%%%%%%%%%%
\newtheorem{claim}{Claim}
\newtheorem{invariant}{Invariant}
\newtheorem{defn}{Definition}
\newtheorem{thm}{Theorem}
\newtheorem{lemma}{Lemma}

\newcommand{\ie}{\emph{i.e.}\xspace}
\newcommand{\eg}{\emph{e.g.}\xspace}

\newcounter{ProblemCounter}
\newenvironment{problem}[1][]
  {\refstepcounter{ProblemCounter}\noindent\textbf{Problem \theProblemCounter{} (#1)}\quad}
  {\newpage}

\newcommand{\answerbelow}{\noindent\makebox[\linewidth]{\rule{\textwidth}{0.4pt}}}


%%%%% Problem-specific Macros %%%%%%%%%%%%%%%%%%%%%%%%%%%%%%%%%%%%%%%%%%%%%%%%%%

\newcommand{\np}{\ensuremath{\mathsf{NP}}\xspace}
\newcommand{\prob}[1]{\ensuremath{\mathsf{#1}}\xspace}
\newcommand{\desc}[1]{\ensuremath{\langle #1 \rangle}}
\newcommand{\Nat}{\ensuremath{\mathbb{N}}\xspace}

%%%%%%%%%%%%%%%%%%%%%%%%%%%%%%%%%%%%%%%%%%%%%%%%%%%%%%%%%%%%%%%%%%%%%%%%%%%%%%%%

\begin{document}

%%%%%%%%%%%%%%%%%%%%%%%%%%%%%%%%%%%%%%%%%%%%%%%%%%%%%%%%%%%%%%%%%%%%%%%%%%%%%%%%

\begin{problem}[Membership]

Consider the following problems and prove that they are in NP.
Be explicit about (a) the type of witnesses of the problem and (b) the verification algorithm for witnesses.
Argue that the verification algorithm operates in polynomial time.
\begin{enumerate}[itemsep=0pt, label=(\alph*)]
  \item \( \prob{COLOR} = \{ \desc{G, k} \;|\; \text{There is a $k$-coloring of graph $G$} \} \).
  \item \( \prob{SCHED} = \{ \desc{J, (\prec), W, k, t} \mid \text{There is a schedule of $J$ jobs on $k$ processors taking time $t$ or less} \} \).
\end{enumerate}

\paragraph{Graph Coloring}
A \( k \)-graph coloring of a graph \( G \) is an assignment of \( k \) colors
to nodes such that no two nodes share both the same color and an edge.

\paragraph{Job Scheduling}
Consider a set of \( J \) jobs equipped with a (1) partial order \( \prec \) on
\( J \) that determines that for each \( J_m, J_n \in J \) whether \( J_m \prec
J_n \) (understood to mean that \( J_m \) must be completed before \( J_n \))
and (2) weight determined by the function \( W : J \rightarrow \mathbb{N} \)
that says how long each job takes to complete. There exists \( k \) distinct
processors on which to run jobs. The goal is to find a scheduling of jobs on
these processors that respects the ordering such that all jobs complete in
target time \( t \) or less.

\answerbelow{}

% FILL IN YOUR ANSWER HERE

\end{problem}

%%%%%%%%%%%%%%%%%%%%%%%%%%%%%%%%%%%%%%%%%%%%%%%%%%%%%%%%%%%%%%%%%%%%%%%%%%%%%%%%

\begin{problem}[The Complexity Strikes Again]

Consider the following combinatorial decision problems:

\paragraph{0-1 Knapsack}
Let \( B \) be a set of objects equipped with two functions:
\begin{itemize}[itemsep=0pt]
  \item A \emph{weight function} \( w : B \rightarrow \Nat \) assigning weights
    to objects.
  \item A \emph{value function} \( v : B \rightarrow \Nat \) assigning values
    to objects.
\end{itemize}
Suppose that we have a knapsack that can hold at most \( W \) weight.  Find a
subset \( B^* \subseteq B \) that can fit in the knapsack and has at least
value \( V \).

\paragraph{Subset Sum}
Let \( S \subseteq \mathbb{N} \) be a (multi-)set of natural numbers.  Find a
subset \( S^* \subseteq S \) such that \( \sum_{s \in S^*} s = t \) for some
target number \( t \).

\begin{enumerate}[label=(\alph*)]
  \item Write down formal language definitions for these two problems:
    \prob{knapsack} and \prob{subsetsum}.
  \item Assume that \prob{subsetsum} is \np-complete.  Prove that
    \prob{knapsack} is \np-complete as well.  Make sure that you explicitly
    justify the complexity of your verifier and the complexity and correctness
    of your constructed mapping function in your proof.
\end{enumerate}

\answerbelow{}

% FILL IN YOUR ANSWER HERE

\end{problem}

\end{document}
