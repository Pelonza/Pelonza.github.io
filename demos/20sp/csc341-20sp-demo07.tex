\documentclass[12pt]{article}

%%%%% Document Information (FILL THIS IN!) %%%%%
\newcommand{\authorname}{(Name)}
\newcommand{\grinnellusername}{(Username)}

\newcommand{\coursenumber}{CSC 341}
\newcommand{\assignmentname}{Demo Exercises \#7}
\newcommand{\assignmentversion}{1}

%%%%% Fonts and Encodings %%%%%%%%%%%%%%%%%%%%%%%%%%%%%%%%%%%%%%%%%%%%%%%%%%%%%%
\usepackage[T1]{fontenc}
\usepackage{libertine}

%%%%% General %%%%%%%%%%%%%%%%%%%%%%%%%%%%%%%%%%%%%%%%%%%%%%%%%%%%%%%%%%%%%%%%%%
\usepackage{amsmath}
\usepackage{amssymb}
\usepackage{enumitem}
\usepackage{fancyvrb}
\usepackage[amsmath, amsthm, thmmarks]{ntheorem}
\usepackage{microtype}
\usepackage{url}
\usepackage[svgnames]{xcolor}
\usepackage{xspace}

\usepackage{tikz}

%%%%% Page Formatting %%%%%%%%%%%%%%%%%%%%%%%%%%%%%%%%%%%%%%%%%%%%%%%%%%%%%%%%%%
\usepackage[
  top=1in,
  bottom=1in,
  left=1in,
  right=1in,
  includefoot,
  paperwidth=8.5in,
  paperheight=11in
]{geometry}

\usepackage{fancyhdr}
\setlength{\headheight}{15.2pt}
\pagestyle{fancy}

\fancyhead{}
\fancyfoot{}
\lhead{\coursenumber{}---\assignmentname{} (ver. \assignmentversion), \authorname{} [\grinnellusername{}]}
\cfoot{\thepage}

%%%%% Basic Macros and Definitions %%%%%%%%%%%%%%%%%%%%%%%%%%%%%%%%%%%%%%%%%%%%%
\newtheorem{claim}{Claim}
\newtheorem{invariant}{Invariant}
\newtheorem{defn}{Definition}
\newtheorem{thm}{Theorem}
\newtheorem{lemma}{Lemma}

\newcommand{\ie}{\emph{i.e.}\xspace}
\newcommand{\eg}{\emph{e.g.}\xspace}
\newcommand{\etc}{\emph{etc.}\xspace}
\newcommand{\hint}[1]{(\emph{Hint}: #1)}

\newcounter{ProblemCounter}
\newenvironment{problem}[1][]
  {\refstepcounter{ProblemCounter}\noindent\textbf{Problem \theProblemCounter{} (#1)}\quad}
  {\newpage}

\newcommand{\answerbelow}{\noindent\makebox[\linewidth]{\rule{\textwidth}{0.4pt}}}


%%%%% Problem-specific Macros %%%%%%%%%%%%%%%%%%%%%%%%%%%%%%%%%%%%%%%%%%%%%%%%%%

\newcommand{\np}{\ensuremath{\mathsf{NP}}\xspace}
\newcommand{\prob}[1]{\ensuremath{\mathsf{#1}}\xspace}
\newcommand{\desc}[1]{\ensuremath{\langle #1 \rangle}}
\newcommand{\Nat}{\ensuremath{\mathbb{N}}\xspace}
\newcommand{\comp}[1]{\ensuremath{\overline{#1}}\xspace}

%%%%%%%%%%%%%%%%%%%%%%%%%%%%%%%%%%%%%%%%%%%%%%%%%%%%%%%%%%%%%%%%%%%%%%%%%%%%%%%%

\begin{document}

\begin{problem}[Honey Bunches of Oats]

\newcommand{\tile}[2]{\ensuremath{\left[\frac{\texttt{#1}}{\texttt{#2}}\right]}}

\begin{enumerate}[itemsep=0pt, label=(\alph*)]
  \item (Sipser 5.3) Find a match in the following instance of the Post
    Correspondence Problem:
    \[
      \left\{ \tile{ab}{abab}, \tile{b}{a}, \tile{aba}{b}, \tile{aa}{a} \right\}.
    \]
  \item (Sipser 5.17) Show that the PCP problem is decidable over the unary
    alphabet \( \Sigma = \{\, a \,\} \).  \hint{With one symbol, you can safely
    consider the \emph{number} of symbols on the top and bottom of each tile.
  From there, think about fractions, \ldots}
\end{enumerate}

\answerbelow

% FILL IN YOUR ANSWER HERE

\end{problem}

%%%%%%%%%%%%%%%%%%%%%%%%%%%%%%%%%%%%%%%%%%%%%%%%%%%%%%%%%%%%%%%%%%%%%%%%%%%%%%%%

\begin{problem}[Scheme Is Real, Right?]

\begin{enumerate}[itemsep=0pt, label=(\alph*)]
  \item Consider the problem of determining whether the branches of an
    arbitrary if-expression in Scheme---\texttt{(if b e1 e2)}---produces only
    one type of value (\eg, only integers or strings).  For example, this
    property is true of \texttt{(if b 0 1)} but false for \texttt{(if b 5
    True)}.  Formulate this problem as a language and show that this problem is
    undecidable.  You may assume the existence of a Scheme function
    \texttt{run-tm} that takes a Turing machine description \( M \) and input
    \( w \) and returns \texttt{\#t} if and only if \( M \) accepts \( w \).
  \item Now consider the fact that a computer program on a 64-bit machine runs
    with \( 2^{64} \) bits of memory.  Show that the halting problem for such
    computer programs is decidable.  Make sure to argue the (a) correctness and
    (b) termination of your algorithm.  \hint{consider the decidability proof
    of \( A_{\mathsf{LBA}} \).}
  \item The answers to the two previous parts seem to be at odds.  In a few
    sentences, describe the critical difference between the two cases that
    makes the first part undecidable but the second part undecidable.
  \item Finally with all this in mind, it seems like that undecidability is
    merely a theoretical device.  However, it does have important practical
    implications!  In a few sentences, describe these practical
    implications---if we feel the need to solve some variant of the halting
    problem for a computational device, what must we keep in mind and how
    should we approach the problem?
\end{enumerate}

\answerbelow

% FILL IN YOUR ANSWER HERE

\end{problem}

%%%%%%%%%%%%%%%%%%%%%%%%%%%%%%%%%%%%%%%%%%%%%%%%%%%%%%%%%%%%%%%%%%%%%%%%%%%%%%%%

\end{document}
