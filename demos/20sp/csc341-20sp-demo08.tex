\skdocumentclass[12pt]{article}

%%%%% Document Information (FILL THIS IN!) %%%%%
\newcommand{\authorname}{(Name)}
\newcommand{\grinnellusername}{(Username)}

\newcommand{\coursenumber}{CSC 341}
\newcommand{\assignmentname}{Demo Exercises \#8}
\newcommand{\assignmentversion}{1}

%%%%% Fonts and Encodings %%%%%%%%%%%%%%%%%%%%%%%%%%%%%%%%%%%%%%%%%%%%%%%%%%%%%%
\usepackage[T1]{fontenc}
\usepackage{libertine}

%%%%% General %%%%%%%%%%%%%%%%%%%%%%%%%%%%%%%%%%%%%%%%%%%%%%%%%%%%%%%%%%%%%%%%%%
\usepackage{amsmath}
\usepackage{amssymb}
\usepackage[shortlabels]{enumitem}
\setlist{nosep}
\usepackage{fancyvrb}
\usepackage[amsmath, amsthm, thmmarks]{ntheorem}
\usepackage{microtype}
\usepackage{url}
\usepackage[svgnames]{xcolor}
\usepackage{xspace}

\usepackage{tikz}

%%%%% Page Formatting %%%%%%%%%%%%%%%%%%%%%%%%%%%%%%%%%%%%%%%%%%%%%%%%%%%%%%%%%%
\usepackage[
  top=1in,
  bottom=1in,
  left=1in,
  right=1in,
  includefoot,
  paperwidth=8.5in,
  paperheight=11in
]{geometry}

\usepackage{fancyhdr}
\setlength{\headheight}{15.2pt}
\pagestyle{fancy}

\fancyhead{}
\fancyfoot{}
\lhead{\coursenumber{}---\assignmentname{} (ver. \assignmentversion), \authorname{} [\grinnellusername{}]}
\cfoot{\thepage}

%%%%% Basic Macros and Definitions %%%%%%%%%%%%%%%%%%%%%%%%%%%%%%%%%%%%%%%%%%%%%
\newtheorem{claim}{Claim}
\newtheorem{invariant}{Invariant}
\newtheorem{defn}{Definition}
\newtheorem{thm}{Theorem}
\newtheorem{lemma}{Lemma}

\newcommand{\ie}{\emph{i.e.}\xspace}
\newcommand{\eg}{\emph{e.g.}\xspace}
\newcommand{\etc}{\emph{etc.}\xspace}
\newcommand{\hint}[1]{(\emph{Hint}: #1)}

\newcounter{ProblemCounter}
\newenvironment{problem}[1][]
  {\refstepcounter{ProblemCounter}\noindent\textbf{Problem \theProblemCounter{} (#1)}\quad}
  {\newpage}

\newcommand{\answerbelow}{\noindent\makebox[\linewidth]{\rule{\textwidth}{0.4pt}}}


%%%%% Problem-specific Macros %%%%%%%%%%%%%%%%%%%%%%%%%%%%%%%%%%%%%%%%%%%%%%%%%%

\newcommand{\NP}{\ensuremath{\mathsf{NP}}\xspace}
\newcommand{\PTIME}{\ensuremath{\mathsf{P}}\xspace}
\newcommand{\BPP}{\ensuremath{\mathsf{BPP}}\xspace}
\newcommand{\Nat}{\mathbb{N}\xspace}

\newcommand{\shrug}{\ensuremath{\stackrel{?}{=}}}

\newcommand{\prob}[1]{\ensuremath{\mathsf{#1}}\xspace}
\newcommand{\desc}[1]{\ensuremath{\langle #1 \rangle}}
\newcommand{\comp}[1]{\ensuremath{\overline{#1}}\xspace}

%%%%%%%%%%%%%%%%%%%%%%%%%%%%%%%%%%%%%%%%%%%%%%%%%%%%%%%%%%%%%%%%%%%%%%%%%%%%%%%%

\begin{document}

\begin{problem}[Mmm, Probably]

Recall that a major theme of this course is understanding the
\emph{relationship} between classes of problems.  Even though we have only
briefly explored probabilistic algorithms, we can think about how their primary
complexity class, \BPP, relates to other classes that we have studied so far.

\begin{enumerate}
  \item Show that \BPP is closed under the union operation:
    \[
      L_1 \cup L_2 = \{\, w \mid w \in L_1 \vee w \in L_2 \,\}.
    \]
    \hint{Adapt the standard union construction for non-deterministic TMs to
      the probabilistic setting. You may invoke the amplification lemma to show
      that your construction's error probabilities fit with the requirements of
      \BPP.}
  \item Recall that the \emph{complement} of a language \( \comp{L} \) is
    the set of strings \emph{not} in the language.  That is:
    \[
      \comp{L} = \{\, w \mid w \notin L \,\}.
    \]
    Show that \BPP is closed under complementation.
  \item A \emph{pseudorandom number generator} (PRNG) is a algorithm that, when
    given an initial value, its \emph{seed}, generates a sequence of numbers
    that appear to be random.  Note that computers do not inherently have a
    source of true randomness (such randomness must be sampled from the outside
    world in some fashion), and so they must resort to PRNGs to generate random
    numbers. Furthermore, call a PRNG \emph{strong} if its outputs are
    ``indistinguishable'' from truly random ones.  That is, the outputs of a
    strong PRNG appear to be chosen uniformly at random.

    Formally, define a PRNG as a deterministic Turing machine \( R_k \) that
    ignores its input and writes a random number in the range of \( 0 \) to \(
    k \) to its tape.  (We will assume for simplicity's sake that \( R_k \) is
    created with an appropriate initial seed value.)  Show that if there exists
    a strong PRNG, then \( \PTIME = \BPP \)\footnote{%
      Do you believe the converse of the statement: if \( \PTIME = \BPP \) then
      there exists strong PRNGs?
    }.
  \item It turns out that \( \PTIME \shrug \BPP \) is still an open problem in
    computer science, like \( \PTIME \shrug \NP \).  However, like \( \PTIME
    \shrug \NP \), most people have strong opinions how \PTIME and \BPP relate.
    Based on your previous result, what do you believe is the most likely
    relationship between \PTIME and \BPP?  What does this relationship imply
    about the ability of randomness to more efficiently solve problems than
    traditional deterministic computation?  Answer these questions in a few
    sentences a piece.
\end{enumerate}

\answerbelow
% FILL IN YOUR ANSWER HERE

\end{problem}

%%%%%%%%%%%%%%%%%%%%%%%%%%%%%%%%%%%%%%%%%%%%%%%%%%%%%%%%%%%%%%%%%%%%%%%%%%%%%%%%

\end{document}
