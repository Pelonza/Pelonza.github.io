\documentclass[12pt]{article}

%%%%% Document Information (FILL THIS IN!) %%%%%
\newcommand{\authorname}{(Name)}
\newcommand{\grinnellusername}{(Username)}

\newcommand{\coursenumber}{CSC 341}
\newcommand{\assignmentname}{Demo Exercises \#5}
\newcommand{\assignmentversion}{2}

%%%%% Fonts and Encodings %%%%%%%%%%%%%%%%%%%%%%%%%%%%%%%%%%%%%%%%%%%%%%%%%%%%%%
\usepackage[T1]{fontenc}
\usepackage{libertine}

%%%%% General %%%%%%%%%%%%%%%%%%%%%%%%%%%%%%%%%%%%%%%%%%%%%%%%%%%%%%%%%%%%%%%%%%
\usepackage{amsmath}
\usepackage{amssymb}
\usepackage{enumitem}
\usepackage{fancyvrb}
\usepackage[amsmath, amsthm, thmmarks]{ntheorem}
\usepackage{microtype}
\usepackage{url}
\usepackage[svgnames]{xcolor}
\usepackage{xspace}

\usepackage{tikz}

%%%%% Page Formatting %%%%%%%%%%%%%%%%%%%%%%%%%%%%%%%%%%%%%%%%%%%%%%%%%%%%%%%%%%
\usepackage[
  top=1in,
  bottom=1in,
  left=1in,
  right=1in,
  includefoot,
  paperwidth=8.5in,
  paperheight=11in
]{geometry}

\usepackage{fancyhdr}
\setlength{\headheight}{15.2pt}
\pagestyle{fancy}

\fancyhead{}
\fancyfoot{}
\lhead{\coursenumber{}---\assignmentname{} (ver. \assignmentversion), \authorname{} [\grinnellusername{}]}
\cfoot{\thepage}

%%%%% Code %%%%%%%%%%%%%%%%%%%%%%%%%%%%%%%%%%%%%%%%%%%%%%%%%%%%%%%%%%%%%%%%%%%%%

\usepackage{listings}
\usepackage{xcolor}

% Adapted from: https://en.wikibooks.org/wiki/LaTeX/Source_Code_Listings
\definecolor{mygreen}{rgb}{0,0.6,0}
\definecolor{mygray}{rgb}{0.5,0.5,0.5}
\definecolor{mymauve}{rgb}{0.58,0,0.82}

\lstset{ 
  backgroundcolor=\color{white},      % choose the background color; you must add \usepackage{color} or \usepackage{xcolor}; should come as last argument
  basicstyle=\ttfamily\footnotesize,  % the size of the fonts that are used for the code
  breakatwhitespace=false,            % sets if automatic breaks should only happen at whitespace
  breaklines=true,                    % sets automatic line breaking
  captionpos=b,                       % sets the caption-position to bottom
  commentstyle=\color{mygreen},       % comment style
  keepspaces=true,                    % keeps spaces in text, useful for keeping indentation of code (possibly needs columns=flexible)
  keywordstyle=\color{blue},          % keyword style
  language=Java,                      % the language of the code
  stringstyle=\color{mymauve},        % string literal style
}

%%%%% Basic Macros and Definitions %%%%%%%%%%%%%%%%%%%%%%%%%%%%%%%%%%%%%%%%%%%%%
\newtheorem{claim}{Claim}
\newtheorem{invariant}{Invariant}
\newtheorem{defn}{Definition}
\newtheorem{thm}{Theorem}
\newtheorem{lemma}{Lemma}

\newcommand{\ie}{\emph{i.e.}\xspace}
\newcommand{\eg}{\emph{e.g.}\xspace}
\newcommand{\hint}[1]{(\emph{Hint}: #1)}

\newcounter{ProblemCounter}
\newenvironment{problem}[1][]
  {\refstepcounter{ProblemCounter}\noindent\textbf{Problem \theProblemCounter{} (#1)}\quad}
  {\newpage}

\newcommand{\answerbelow}{\noindent\makebox[\linewidth]{\rule{\textwidth}{0.4pt}}}


%%%%% Problem-specific Macros %%%%%%%%%%%%%%%%%%%%%%%%%%%%%%%%%%%%%%%%%%%%%%%%%%

\newcommand{\np}{\ensuremath{\mathsf{NP}}\xspace}
\newcommand{\prob}[1]{\ensuremath{\mathsf{#1}}\xspace}
\newcommand{\desc}[1]{\ensuremath{\langle #1 \rangle}}
\newcommand{\Nat}{\ensuremath{\mathbb{N}}\xspace}
\newcommand{\comp}[1]{\ensuremath{\overline{#1}}\xspace}

%%%%%%%%%%%%%%%%%%%%%%%%%%%%%%%%%%%%%%%%%%%%%%%%%%%%%%%%%%%%%%%%%%%%%%%%%%%%%%%%

\begin{document}

\begin{problem}[Satisfying Variations]

For this problem, we'll explore variations of the 3-SAT problem, generalized to
the \( k \)-SAT problem:
\begin{gather*}
  \prob{kSAT} = \{ \desc{\phi} \mid \text{\( \phi \) is a satisfiable \( k \)cnf-formula} \}.
\end{gather*}

\begin{enumerate}[itemsep=0pt, label=(\alph*)]
  \item Suppose that you have a 2cnf-formula clause of the form \( (x_1 \vee
    x_2) \). Describe how you would turn this clause into a logically equivalent
    3cnf-formula clause. Argue in a sentence or two why your transformation is
    correct. \item With this fact, prove that \( \prob{2SAT} \leq_p \prob{3SAT}
    \).
  \item Suppose that you have a 4cnf-formula clause of the form \( (x_1 \vee x_2
    \vee x_3 \vee x_4 ) \). Describe how you would turn this clause into a
    logically equivalent 3cnf-formula clause.  Argue in a sentence or two why
    your transformation is correct. \hint{you will want to split up the 4cnf
      clause into, \eg, \( (x_1 \vee x_2 \vee \ldots) \wedge (\ldots \vee x_3
      \vee x_4) \). How do you complete these two 3cnf clauses so that only one
      of \( x_1, \ldots, x_4 \) is necessary to be true to make \emph{both}
      clauses true? You should consider making a new variable distinct from \(
      x_1, \ldots, x_4 \) to help you with this transformation.}
  \item With this fact, prove that \( \prob{4SAT} \leq_p \prob{3SAT} \).
  \item Finally, we can generalize \( k \)-SAT to any combination of sizes of
    cnf-formula as follows:
    \[
      \prob{CSAT} = \{ \desc{\phi} \mid \text{ \( \phi \) is a satisfiable
        formula in CNF form } \}.
    \]
    Show that \( \prob{CSAT} \leq_p \prob{3SAT} \).  \hint{each sub-formula of
    \( \phi \) can have arbitrary size.  Make sure that your reduction covers
    \emph{every} possible size of formula.}
\end{enumerate}

\answerbelow{}

% FILL IN YOUR ANSWER HERE

\end{problem}

%%%%%%%%%%%%%%%%%%%%%%%%%%%%%%%%%%%%%%%%%%%%%%%%%%%%%%%%%%%%%%%%%%%%%%%%%%%%%%%%

\begin{problem}[Color Me Badd]

\tikzset{gnode/.style={circle,fill=white!20,draw,minimum size=1cm,inner sep=0pt},}

A \( k \)-coloring of a graph \( G \) is an assignment of \( k \) distinct
colors to the nodes of \( G \) such that no two nodes that share an edge also
share the same color.  In this problem, we will show that: \( \prob{3COLOR} =
\{ \desc{G} \mid \text{\( G \) is a graph at \( G \) has a 3-coloring} \} \) is
\np-complete.

\begin{enumerate}[itemsep=0pt, label=(\alph*)]
  \item Prove that \( \prob{3COLOR} \in \np \).
  \item To prove that \( \prob{3COLOR} \) is \np-hard, we will reduce from
    \( \prob{3SAT} \).  Our mapping function takes as input a 3cnf-formula \(
    \phi \) and produces a graph \( G \) as output with the property:
    \[
      \text{\( \phi \) is satisfiable} \Leftrightarrow \text{\( G \) has a 3-coloring}.
    \]
    The core of this construction is a \emph{color gadget} which is a sub-graph
    that forces a particular coloring of its nodes:
    \begin{center}
      \begin{tikzpicture}[node distance=1.3cm, >=stealth, bend angle=45, auto]
        \node [gnode] (1) at (0, 0) {T};
        \node [gnode] (2) at (1, 2) {F};
        \node [gnode] (3) at (2, 0) {B};
        \draw (1) -- (2) -- (3) -- (1);
      \end{tikzpicture}
    \end{center}
    Note that the color gadget is a complete 3-graph.  Because of this, each
    node must be a distinct color.  Here, we will use the colors \( T \), \( F
    \), and \( B \).  The first two colors will correspond to the values true
    and false respectively.  The final color is a ``base'' color that we
    introduce merely to fulfill the requirement that the graph coloring must
    employ three colors.  With the color gadget, we are able to \emph{force}
    nodes in our overall construction to take on particular colors.

    Recall in our mapping function that we must map objects of our input
    domain---variables, clauses, and assignments of \( \phi \)---to objects in
    our output domain---nodes, edges, and colorings of \( G \).  First we will
    map the variables to nodes and assignments to colorings.  For each \( x_i
    \), we will create two nodes labeled \( x_i \) and \( \comp{x_i} \).
    Create a subgraph with these two nodes and the color gadget to force
    exactly one of the two nodes to be colored \( T \) and the other \( F \).
    Argue in a few sentences why your constructed subgraph correctly implements
    this behavior.
  \item Next, we will describe how to map the input clauses onto the output
    graph.  Suppose that we have two variable nodes \( x_i \) and \( x_j \) and
    we want to compute the ``or'' of the two nodes.  Create a subgraph with
    these two nodes with the property that one of the nodes of this subgraph,
    the ``output'' node, *can* be colored \( T \) if at least one of \( x_i \)
    and \( x_j \) are colored \( T \).  Give an example of your ``or''-gadget
    and how it may output \( T \) when given two variables, exactly one of
    which is colored \( T \).
  \item Extend the ``or''-gadget from the previous part to three variable nodes
    \( x_i \), \( x_j \), and \( x_k \).  Furthermore, force the output node of
    this gadget to be colored \( T \).  Argue in a few sentences why the clause
    gadget outputs \( T \) if and only if at least one of \( x_i \), \( x_j \),
    and \( x_k \) is true. \hint{Note that ``or'' gadget from the previous part
    may not have the ``if and only if'' behavior that we need.  However, this
    extended gadget should have that behavior, but its a surprisingly subtle
    argument! Use the color gadget to force colorings as necessary and argue
    why certain ``bad'' colorings present in the basic ``or'' gadget are not
    possible with the extended gadget.}
  \item Describe the complete mapping function \( m : \phi \rightarrow G \) for
    this reduction, combining all of the gadgets from the previous parts.
  \item As an example, give the output of your mapping function for the
    following 3-sat formula (from figure 7.33 of Sipser):
    \[
      \phi = (x_1 \vee x_1 \vee x_2) \wedge (\comp{x_1} \vee \comp{x_2} \vee
        \comp{x_2}) \wedge (\comp{x_1} \vee x_2 \vee x_2)
    \]
  \item Finally, we'll prove the correctness of our construction.  First, prove
    the ``if'' direction: if \( \phi \) is satisfiable, then \( m(\phi) \) is
    3-colorable.  You may appeal to your previous arguments for the correctness
    of the sub-components of the construction as necessary.
  \item Prove the ``only if'' direction of the claim: if \( m(\phi) = G \) and
    \( G \) is 3-colorable, then \( \phi \) is satisfiable.  You may appeal to
    your previous arguments for the correctness of the sub-components of the
    construction as necessary.
\end{enumerate}

\answerbelow{}

% FILL IN YOUR ANSWER HERE

\end{problem}

%%%%%%%%%%%%%%%%%%%%%%%%%%%%%%%%%%%%%%%%%%%%%%%%%%%%%%%%%%%%%%%%%%%%%%%%%%%%%%%%

\end{document}
