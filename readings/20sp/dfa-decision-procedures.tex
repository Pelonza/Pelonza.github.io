\documentclass[11pt]{book}

%%%%% Handout Definitions %%%%%
\setcounter{chapter}{6}
\newcommand{\docclass}{CSC 341\xspace}
\newcommand{\doctitle}{DFA Decision Procedures}
\newcommand{\docauthor}{Peter-Michael Osera}

\usepackage{reading}

%%%%% Drawing %%%%%
\usepackage{tikz}
\usetikzlibrary{automata, positioning}

%%%%%%%%%%%%%%%%%%%%%%%%%%%%%%%%%%%%%%%%%%%%%%%%%%%%%%%%%%%%%%%%%%%%%%%%%%%%%%%%

\begin{document}

\section*{DFA Decision Procedures}

When studying models of computation, we are primarily interested in two kinds of results:
\begin{itemize}[itemsep=0pt]
  \item \emph{Closure Properties} (Sipser §1.2)---given an operation over a particular sort of language (\eg, the regular languages), does the operation preserve the sort of the input languages?
    For example, the result of concatenating two regular languages is always a regular language.
  \item \emph{Decision Procedures}---given a machine (\eg, a DFA), can we determine if a proposition about that machine holds?
    For example, is a given DFA minimal with respect to the number of states that it contains?
\end{itemize}
Here, we are concerned with the second kind of results.
In particular, we will study the following problems:
\begin{itemize}[itemsep=0pt]
  \item Emptiness: given a DFA (equivalently, an NFA or regular expression), is its language empty?
  \item Acceptance: given a DFA (equivalently, an NFA or regular expression) and an input string, does the DFA recognize that input string?
  \item Minimization: given a DFA, is it minimal with respect to the number of states that it contains?
  \item Equivalence: given two DFAs (equivalently, NFAs or regular expressions), are the DFAs \emph{equivalent}?
    That is, do they accept the same set of strings and reject the same set of strings?
\end{itemize}

\subsection*{Emptiness and Acceptance}

The language of a DFA $D$ is \emph{empty} if it is the empty set, \ie, $L(D) = \emptyset$.
The emptiness problem consists of an input DFA $D$ and determining if $L(D) = \emptyset$.
A na\"{i}ve strategy for solving this problem is running $D$ on every possible string $w \in \Sigma^*$, returning false if any such $w$ is rejected by $D$.
This does not work because there are an infinite number of such strings---we never have the opportunity to return true!

Instead, we must determine whether $L(D) = \emptyset$ by inspecting the structure of $D$.
To get a feel of how to check this, consider the following DFA:

IMG-01 A simple DFA.

The language of this DFA is non-empty.
It at least accepts the string $10$.
In contrast, consider the following DFA:

IMG-02 A simple DFA with no accepting states.

This DFA is identical to the first, except that it does not have any accepting states.
Because of this, no string will ever be accepted by the DFA and so its language is empty.

In light of this, it is clear that the language of any DFA whose set of accept states is empty will be empty as well.
However, there is one other case we must consider.

IMG-03 A DFA with unreachable accepting states.

Here, the DFA has an accepting state, $q_1$.
However, there is no way to reach that accepting state from the start state.

To formalize this idea, let's define the notion of \emph{reachability} in a DFA.
\begin{defn}
  In a DFA $D = (\Sigma, Q, \delta, q_0, F)$, a state $q' \in Q$ is \emph{reachable} from another state $q \in Q$ if there exists a $w \in \Sigma$ such that $\delta(\delta(\delta(q, w_0), w_1), \ldots), w_n) = q'$ where $w = w_0 w_1 \cdots w_n$.
\end{defn}
Intuitively, a state $q'$ is reachable from a state $q$ if there exists a series of characters we can read to go from $q$ to $q'$ according to the transition function $\delta$.
In the case where $q = q_0$, then the series of characters we read correspond to strings accepted by the DFA.

Thus, the emptiness problem is a matter of determining \emph{reachability} in a DFA.
We can define the solution to the emptiness problem as follows:
\begin{thm}
  Given a DFA $D = (\Sigma, Q, \delta, q_0, F)$, $L(D) = \emptyset$ if and only if no $q_f \in F$ is reachable from $q_0$.
\end{thm}
\begin{proof}
  In both directions of the biconditional, correctness follows from the fact that the definition of reachability is analogous to the definition of acceptance of a DFA: ``There exists a chain of characters that take the machine from the start state to an accepting state.''
\end{proof}
Note that we don't need to enumerate all possible strings to test reachability.
It is sufficient to simply perform a breadth- or depth-first search starting from $q_0$ to see if we are able to reach an accepting state.

Likewise, the acceptance problem considers a DFA $D$ and a string $w$ and asks whether $w \in L(D)$.
This also corresponds to reachability:

\begin{thm}
  Given a DFA $D = (\Sigma, Q, \delta, q_0, F)$ and string $w \in \Sigma^*$, $w \in L(D)$ if and only if $q_f \in F$ is reachable from $q_0$ using $w$.
\end{thm}
\begin{proof}
  Again, both directions of the biconditional are immediate from the definition of reachability and DFA acceptance.
\end{proof}

Here, we simply need to hand-simulate execution of $D$ on $w$ to determine acceptance.

\subsection*{Minimization and Equivalence}

Consider the following DFA:

IMG-4 A DFA that recognizes $(0 \cup 1)01$.

The language of this DFA are the strings corresponding to the following regular expression: $(0 \cup 1)01$.
However, this is a simpler DFA that recognizes the same language:

IMG-5 A simpler DFA that recognizes $(0 \cup 1)01$.

This DFA is simpler in the sense that it has one fewer states than the original DFA.
We could try to reduce this DFA further, but it turns out that this DFA is the \emph{smallest} DFA recognizing the language $(0 \cup 1)01$.
The process of taking a DFA that recognizes language $L$ and finding a corresponding DFA with less states that also recognizes $L$ is called \emph{minimization}.

The intuition behind minimizing a DFA is that some states in a non-minimal DFA exhibit the \emph{same behavior}.
For example, consider $q_1$ and $q_2$ in the original DFA.
Even though they are different states, you can see that for any string $w$ that simulating the machine on input $w$ starting at either $q_1$ or $q_2$ results in the same result: either both cases accept the string or reject the string.
This is because for any character, $q_1$ and $q_2$ both move to the same states ($q_3$ on a $0$ and $q_5$ on a $1$).
In this sense, $q_1$ and $q_2$ are \emph{indistinguishable}---we can combine them without any change in the behavior of the DFA.
Formally, we define two states to be distinguishable as follows:
\begin{defn}
  In a DFA $D = (\Sigma, Q, \delta, q_0, F)$, states $p, q \in Q$ are \emph{distinguishable} if there exists a string $w \in \Sigma^*$ such that $\delta^*(p, w) = p'$ and $\delta^*(q, w) = q'$ and either $p' \in F, q' \not\in F$ or $p' \not\in F, q' \in F$.
\end{defn}
Indistinguishability is defined similarly:
\begin{defn}
  In a DFA $D = (\Sigma, Q, \delta, q_0, F)$, states $p, q \in Q$ are \emph{indistinguishable} if for all $w \in \Sigma^*$, $\delta^*(p, w) = p'$ and $\delta^*(q, w) = q'$ and either $p' \in F, q' \in F$ or $p' \not\in F, q' \not\in F$.
\end{defn}

With this in mind, the \emph{DFA minimization process} works in three steps:

\begin{enumerate}[itemsep=0pt]
  \item First, remove from the DFA any states that are \emph{unreachable} from the start state (using the definition of reachability from the previous section).
  \item Discover pairs of states in the DFA that are indistinguishable.
  \item Combine indistinguishable states to achieve a final, \emph{minimal} DFA.
\end{enumerate}

To discover pairs of indistinguishable states, we use a \emph{table-filling algorithm} that tracks which pairs of states are indistinguishable.
We repeatedly refine the table, discovering new states that are indistinguishable until we reach a point where we make no new discoveries.
The final table denotes which states are indistinguishable and thus combinable in the final step of the algorithm.

As an example, consider running the table-filling algorithm on the original DFA.
Our table has the following shape:

\vspace{1em}

\begin{tabular}{|c|c|c|c|c|c|c|}
  \hline
  $q_0$ & $\times$ & $\times$ & $\times$ & $\times$ & $\times$ & $\times$ \\
  \hline
  $q_1$ &          & $\times$ & $\times$ & $\times$ & $\times$ & $\times$ \\
  \hline
  $q_2$ &          &          & $\times$ & $\times$ & $\times$ & $\times$ \\
  \hline
  $q_3$ &          &          &          & $\times$ & $\times$ & $\times$ \\
  \hline
  $q_4$ &          &          &          &          & $\times$ & $\times$ \\
  \hline
  $q_5$ &          &          &          &          &          & $\times$ \\
  \hline
        & $q_0$    & $q_1$    & $q_2$    & $q_3$    & $q_4$    & $q_5$    \\
  \hline
\end{tabular}

\vspace{1em}

An entry in the table is marked with a circle ($\circ$) if we have discovered that the corresponding pair of states are \emph{distinguishable}.
If the entry is empty, we say that the states are \emph{indistinguishable}.
Entries marked with a cross ($\times$) are ignored as they are redundant (since the ordering does not matter in these pairs of states).

First, from our definition of distinguished states, we see that any accepting state is distinguished from a non-accepting state.
We therefore mark any pair of states that contains $q_4$ as distinguished:

\vspace{1em}

\begin{tabular}{|c|c|c|c|c|c|c|}
  \hline
  $q_0$ & $\times$ & $\times$ & $\times$ & $\times$ & $\times$ & $\times$ \\
  \hline
  $q_1$ &          & $\times$ & $\times$ & $\times$ & $\times$ & $\times$ \\
  \hline
  $q_2$ &          &          & $\times$ & $\times$ & $\times$ & $\times$ \\
  \hline
  $q_3$ &          &          &          & $\times$ & $\times$ & $\times$ \\
  \hline
  $q_4$ & $\circ$  & $\circ$  & $\circ$  & $\circ$  & $\times$ & $\times$ \\
  \hline
  $q_5$ &          &          &          &          & $\circ$  & $\times$ \\
  \hline
        & $q_0$    & $q_1$    & $q_2$    & $q_3$    & $q_4$    & $q_5$    \\
  \hline
\end{tabular}

\vspace{1em}

Now, we repeat the following process until we no longer change the table:
\begin{enumerate}[itemsep=0pt]
  \item For each pair of states $(p, q)$ and for each possible character $a \in \Sigma$, consider the new pair of states $(\delta(p, a), \delta(p, q))$.
  \item If $(p, q)$ is undistinguished, but $(\delta(p, a), \delta(p, q))$ is distinguished, then mark $(p, q)$ as distinguished.
\end{enumerate}

For example, on the first iteration of our process, we first consider the pair $(q_1, q_0)$ (the top-left corner of the table although the order does not matter) and the possible transitions from it:
\begin{itemize}[itemsep=0pt]
  \item On input $0$, $(\delta(q_1, 0), \delta(q_0, 0)) = (q_3, q_1)$.
  \item On input $1$, $(\delta(q_1, 1), \delta(q_0, 1)) = (q_5, q_2)$.
\end{itemize}
We note that $(q_3, q_1)$ and $(q_5, q_2)$ are not marked (\ie, are considered undistinguished), so we do not change the entry for $(q_1, q_0)$.

In contrast, consider the pair $(q_2, q_3)$:
\begin{itemize}[itemsep=0pt]
  \item On input $0$, $(\delta(q_2, 0), \delta(q_3, 0)) = (q_3, q_5)$.
  \item On input $1$, $(\delta(q_2, 1), \delta(q_3, 1)) = (q_5, q_4)$.
\end{itemize}
While $(q_3, q_5)$ is not marked, $(q_5, q_4)$ is marked, so we mark $(q_2, q_3)$.

Continuing this process, we arrive at the following updated table

\vspace{1em}

\begin{tabular}{|c|c|c|c|c|c|c|}
  \hline
  $q_0$ & $\times$ & $\times$ & $\times$ & $\times$ & $\times$ & $\times$ \\
  \hline
  $q_1$ &          & $\times$ & $\times$ & $\times$ & $\times$ & $\times$ \\
  \hline
  $q_2$ &          &          & $\times$ & $\times$ & $\times$ & $\times$ \\
  \hline
  $q_3$ & $\circ$  & $\circ$  & $\circ$  & $\times$ & $\times$ & $\times$ \\
  \hline
  $q_4$ & $\circ$  & $\circ$  & $\circ$  & $\circ$  & $\times$ & $\times$ \\
  \hline
  $q_5$ &          &          &          & $\circ$  & $\circ$  & $\times$ \\
  \hline
        & $q_0$    & $q_1$    & $q_2$    & $q_3$    & $q_4$    & $q_5$    \\
  \hline
\end{tabular}

\vspace{1em}

Note that because the only pairs of states involving $q_4$ were marked as distinguished initially, then only pairs of states that transitioned into a pair containing $q_4$ are marked as distinguished in this first round.
The only such state is $q_3$, so only pairs of states involving $q_3$ were marked in this round.

Since we modified the table in the first round, we repeat the process again for all the unmarked states.
On the next iteration, the updated table is:

\vspace{1em}

\begin{tabular}{|c|c|c|c|c|c|c|}
  \hline
  $q_0$ & $\times$ & $\times$ & $\times$ & $\times$ & $\times$ & $\times$ \\
  \hline
  $q_1$ & $\circ$  & $\times$ & $\times$ & $\times$ & $\times$ & $\times$ \\
  \hline
  $q_2$ & $\circ$  &          & $\times$ & $\times$ & $\times$ & $\times$ \\
  \hline
  $q_3$ & $\circ$  & $\circ$  & $\circ$  & $\times$ & $\times$ & $\times$ \\
  \hline
  $q_4$ & $\circ$  & $\circ$  & $\circ$  & $\circ$  & $\times$ & $\times$ \\
  \hline
  $q_5$ &          & $\circ$  & $\circ$  & $\circ$  & $\circ$  & $\times$ \\
  \hline
        & $q_0$    & $q_1$    & $q_2$    & $q_3$    & $q_4$    & $q_5$    \\
  \hline
\end{tabular}

\vspace{1em}

And on the final iteration, the updated table is:

\vspace{1em}

\begin{tabular}{|c|c|c|c|c|c|c|}
  \hline
  $q_0$ & $\times$ & $\times$ & $\times$ & $\times$ & $\times$ & $\times$ \\
  \hline
  $q_1$ & $\circ$  & $\times$ & $\times$ & $\times$ & $\times$ & $\times$ \\
  \hline
  $q_2$ & $\circ$  &          & $\times$ & $\times$ & $\times$ & $\times$ \\
  \hline
  $q_3$ & $\circ$  & $\circ$  & $\circ$  & $\times$ & $\times$ & $\times$ \\
  \hline
  $q_4$ & $\circ$  & $\circ$  & $\circ$  & $\circ$  & $\times$ & $\times$ \\
  \hline
  $q_5$ & $\circ$  & $\circ$  & $\circ$  & $\circ$  & $\circ$  & $\times$ \\
  \hline
        & $q_0$    & $q_1$    & $q_2$    & $q_3$    & $q_4$    & $q_5$    \\
  \hline
\end{tabular}

\vspace{1em}

(I recommend stepping through the algorithm and checking your work with the tables above.)
In the final iteration, the pair $(q_2, q_1)$ is left unmarked.
To see why we'll never mark $(q_2, q_1)$, look at its transitions:
\begin{itemize}[itemsep=0pt]
  \item On input $0$, $(\delta(q_2, 0), \delta(q_1, 0)) = (q_3, q_3)$.
  \item On input $1$, $(\delta(q_2, 1), \delta(q_1, 1)) = (q_5, q_5)$.
\end{itemize}
And note that the pairs $(q_3, q_3)$ and $(q_5, q_5)$ are not marked as distinguished.

Because of this, successive iterations of the process do not change the table.
Therefore, this is the final result and we note that $q_2$ and $q_1$ are indistinguishable.
Collapsing these two states into a single state results in the minimized DFA given above.

A surprising but important fact about our minimization algorithm is that it produces a \emph{unique} DFA.
\begin{thm}
  Consider a DFA $D$.
  There exists a unique DFA $D' = (\Sigma, Q', \delta', q_0', F')$ (up to renaming of states) such that $L(D) = L(D')$ and for any other DFA $D'' = (\Sigma, Q'', \delta'', q_0'', F'')$ such that $L(D) = L(D'')$ and $|Q'| < |Q''|$.
  The table-filling algorithm above derives this $D'$ for some given DFA $D$.
\end{thm}
We'll withhold the proof of this claim until later when we have additional machinery to reason about the \emph{equivalence classes} of states discovered by our algorithm.

With this theorem, we can tackle the closely-related problem of DFA \emph{equivalence}:
\begin{defn}
  Let $D_1$ and $D_2$ be DFAs.
  $D_1$ is \emph{equivalent} to $D_2$, written $D_1 \equiv D_2$, if $L(D_1) = L(D_2)$.
  That is $D_1$ accepts the same strings as $D_2$, and $D_1$ rejects the same strings as $D_2$.
\end{defn}
Luckily, because the minimal DFA $D'$ for some DFA $D$ is unique, we can use the following procedure to see if two DFAs $D_1$ and $D_2$ are equivalent:
\begin{enumerate}
  \item Run the minimization procedure on $D_1$ and $D_2$ to produce minimal DFAs $D_1'$ and $D_2'$.
  \item If $D_1'$ and $D_2'$ are identical (up to renaming of states), then $D_1 \equiv D_2$.
\end{enumerate}

\end{document}
