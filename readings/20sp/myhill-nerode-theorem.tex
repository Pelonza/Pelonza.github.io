\documentclass[11pt]{book}

%%%%% Handout Definitions %%%%%
\setcounter{chapter}{7}
\newcommand{\docclass}{CSC 341\xspace}
\newcommand{\doctitle}{The Myhill-Nerode Theorem}
\newcommand{\docauthor}{Peter-Michael Osera}

\usepackage{reading}

%%%%% Drawing %%%%%
\usepackage{tikz}
\usetikzlibrary{automata, positioning}

\begin{document}

\section*{The Myhill-Nerode Theorem}

Recall in our discussion of DFA equivalence, we were concerned with collapsing pairs of indistinguishable states.
The canonical example of this were states of the following form:

PNG-01

Assuming that $\Sigma = \{ 0, 1 \}$, we see that $q_0$ and $q_1$ go to the same state on every input.
If we arrive at either $q_0$ or $q_1$, our machine will accept and reject the same set of strings according to whatever transitions follow from $q_2$.
Therefore, it is safe to collapse these two states into one:

PNG-02

Let's formalize this notion of indistinguishableness.
By doing this, we will arrive at the \emph{Myhill-Nerode theorem}, a result that describes a fundamental property of regular languages.
We can use this result to justify our equivalence algorithm as well as show that a language is regular or non-regular.

\subsection*{Distinguishing Extensions and Equivalences}

Consider the following DFA slightly modified from our original example.

PNG-03

Imagine two separate executions of this DFA.
To get to $q_0$ in the first execution and $q_1$ in the second execution, we would need to first read in some strings $x \in \Sigma^*$ and $y \in \Sigma^*$, respectively.

Now, imagine the additional transitions that result from reading in an additional character in both of these executions.
If in both executions we read a $0$, then we are in the situation where we converge on the same state.
Both executions will have the behavior at this point---rejection, since they are now both stuck in $q_2$.
In contrast, if we read a $1$, then we are in the situation where the first execution (originally in $q_0$) will always accept by moving to $q_3$ whereas the second execution (originally in $q_1$) will always reject by moving to $q_4$.

Here, the additional character $0$ or $1$ that is read is called an \emph{extension} of the strings $x$ and $y$.
$1$ is called a \emph{distinguishing} extension to the strings $x$ and $y$ with respect to the DFA: by reading in a $1$, the behavior of the executions starting in $q_0$ and $q_1$ differ.
In contrast, $0$ is a \emph{nondistinguishing} extension to $x$ and $y$ because in both executions, we would accept.

We can generalize this notion of distinguishability to arbitrary strings $z \in \Sigma^*$, rather than single characters as follows:
\begin{defn}
  Consider a language $L \subseteq \Sigma^*$ and strings $x, y \in \Sigma^*$.
  A string $z \in \Sigma^*$ is a \emph{distinguishing extension} of the strings $x$ and $y$ if exactly one of $xz$ and $yz$ are in $L$.
\end{defn}
\begin{defn}
  Consider a language $L \subseteq \Sigma^*$ and strings $x, y \in \Sigma^*$.
  Two strings $x$ and $y$ are \emph{indistinguishable}, written $x \equiv_L y$ if there does not exist a $z \in \Sigma^*$ that is a distinguishing extension of $x$ and $y$ with respect to $L$.
\end{defn}

This relation, $\equiv_L$, forms an \emph{equivalence relation} between strings in $\Sigma^*$.
Recall that an equivalence is a particular relation satisfying three properties:
\begin{enumerate}[itemsep=0pt]
  \item Reflexivity ($x \equiv_L x$).
  \item Symmetry ($x \equiv_L y \Rightarrow y \equiv_L x$).
  \item Transitivity ($x \equiv_L y \wedge y \equiv_L z \Rightarrow x \equiv_L z$).
\end{enumerate}
Clearly the first two properties hold---identical strings have the same behavior with respect to extension and the relation does not favor its left- or right-hand sides.
To see that the third property holds, consider some extension $w \in \Sigma^*$ to $x$, $y$, and $z$.
By our premises, we know that $xw$ and $yw$ have the same accept/reject behavior and $yw$ and $zw$ have the same accept/reject behavior.
To conclude that $xw$ and $zw$ have the same behavior, we note that they must both have the same behavior as $yw$.

\subsection*{The Theorem}

With an equivalence relation, we can form \emph{equivalence classes} between the members it relates.
Recall that an equivalence class is a set of elements closed under the equivalence relation.
That is, we establish that two elements are equivalent, $x \equiv_L y$, and then use the properties of the equivalence relation to find additional equivalent elements until there are no more to add to the set.

The Myhill-Nerode theorem characterizes the regular languages in terms of the equivalence classes generated by $\equiv_L$.
\begin{thm}
A language $L$ is regular if and only if $\equiv_L$ has a finite number of equivalence classes.
Furthermore, the number of equivalence classes corresponds to the number of states in the minimal DFA that recognizes $L$.
\end{thm}
\begin{proof}
  In the ``if'' direction, we have a regular language $L$ and from it must deduce that $\equiv_L$ has a finite number of equivalence classes.
  Consider a DFA $D = (\Sigma, Q, \delta, q_0, F)$ that recognizes $L$.
  Define the following family of sets of strings for each $q_i \in Q$:
  \[
    W_i = \{ w \in \Sigma^* \;|\; q_i \in Q, \delta^*(q_0, w) = q_i \}.
  \]
  That is $W_i$ is the set of strings that cause $D$ to transition from $q_0$ to $q_i$.
  Note that every string belongs to exactly one of these $W_i$, \ie, each $w \in \Sigma^*$ sends $D$ to a single $q_i$ by the definition of the transition function $\delta$.

  Now, consider an arbitrary $q_i \in Q$, its related $W_i$, and any two $x, y \in W_i$.
  Since $\delta^*(x, q_0) = \delta^*(y, q_0) = q_i$ ($x$ and $y$ arrive at the same state $q_i$ in $D$), then we know there is no distinguishing extension $z \in \Sigma^*$ for the strings $x$ and $y$.
  Therefore, $x \equiv_L y$ and we can therefore conclude that $W_i$ is a subset of some equivalence class of $\equiv_L$.
  From this and the fact that every possible string belongs to exactly one of these sets, we know that the size of the family of sets $W_i$, \ie, $|Q|$, is an upper bound on the total number of equivalence classes of $\equiv_L$.
  Because $|Q|$ has finite size, we can finally conclude that the number of equivalence classes is finite.
  (Note that we are unable to conclude that each $W_i$ is \emph{equal} to some equivalence class of $\equiv_L$, rather than a subset, because multiple $W_i$ might map onto the same equivalence class.)

  In the ``only if`` direction, we know that $\equiv_L$ has a finite number of equivalence classes and from there must deduce that $L$ is regular.
  To do this, we construct a DFA $D = (\Sigma, Q, \delta, q_0, F)$ from $\equiv_L$ where $L(D) = L$.
  Let $W_i$ denote the $i$th equivalence class of $\equiv_L$.
  \begin{itemize}[itemsep=0pt]
    \item $Q = \{ q_i \;|\; W_i \subseteq \Sigma^* \}$.
    \item $\delta(q_i, a) = q_j$ where $W_i$ corresponds to the equivalence class for the string $x \in W_i$, and $W_j$ corresponds to the equivalence class for the string $xa$.
      Note that it doesn't matter which $x$ is chosen; by the definition of $\equiv_L$, any extension of $x, y \in W_i$ has the same behavior, so choosing either $x$ and $y$ results in $W_j$.
    \item $q_0 = q_i$ where $\epsilon \in W_i$, \ie, we start in the equivalence class that contains the empty string.
    \item $F = \{ q_i \;|\; w \in W_i \wedge w \in L \}$.
  \end{itemize}

  Finally, note that in the ``if'' direction, we showed the sets of strings derived from a DFA by considering distinguishing extensions is a subset of an equivalence class in $\equiv_L$.
  Furthermore, there may be more such sets than there are equivalence classes.
  The ``only if'' direction shows that we can directly derive a DFA from the equivalence classes of $\equiv_L$.
  From these things, we can conclude that the DFA derived from the equivalence classes of $\equiv_L$ is the \emph{minimal} DFA for the language $L$.
\end{proof}

\subsection*{Applications}

The Myhill-Nerode theorem establishes a direct correspondence between the equivalence classes of $\equiv_L$ and the states of the (minimal) DFA that recognizes $L$.
As an immediate corollary, the table-filling algorithm for minimization that we discussed in the previous reading computes the equivalence classes of $\equiv_L$ and therefore, is minimal.

In addition to this, we can use the Myhill-Nerode theorem to establish whether a language $L$ is regular.
To do this, we either construct the finite equivalence classes of $\equiv_L$ (to show regularity) or show that the equivalence classes of $\equiv_L$ is infinite (to show non-regularity).

\begin{claim}
  Consider the language $L = \{ w \;|\; \text{$w$ contains an even number of $0$s} \}$.
  $L$ is regular.
\end{claim}
\begin{proof}
  Rather than building a DFA, we can show this directly using the Myhill-Nerode theorem.
  We enumerate possible strings, using distinguishing extensions to group them into equivalence classes:
  \begin{enumerate}
    \item The strings $\epsilon, 1, 00, 010, \ldots$: these strings have the same acceptance/rejection behavior on all possible extensions, \eg, $\epsilon$ (accept), $0$ (reject), $1$ (accept), \ldots
    \item The strings $0, 01, 10, 110, \ldots$: these strings have the same acceptance/rejection behavior on all possible extensions, \eg, $\epsilon$ (reject), $0$ (accept), $0$ (reject), \ldots
  \end{enumerate}
  Formally, we define two sets of strings:
  \begin{itemize}
    \item $W_1 = \{ w \;|\; \text{$w$ has an even number of 0s} \}$.
    \item $W_2 = \{ w \;|\; \text{$w$ has an odd number of 0s} \}$.
  \end{itemize}
  These two sets are complete---they encompass all strings in $\Sigma^*$.
  Furthermore, any $w \in \Sigma^*$ is in exactly one of $W_1$ and $W_2$.
  Finally, each set is an equivalence class of $\equiv_L$—any extension of pairs of strings in one of the $W_i$ result in identical acceptance/rejection behavior with respect to $L$.
  Thus, we can conclude $W_1$ and $W_2$ are the equivalence classes of $\equiv_L$ and thus $L$ is regular.

  Note that these two classes correspond to the following minimal DFA:

PNG-04

which we recognize as the correct DFA for $L$.
\end{proof}

\begin{claim}
  Consider the language $L = \{ 0^n1^n \;|\; n \geq 0 \}$.
  $L$ is not regular.
\end{claim}
\begin{proof}
  Rather than using the pumping lemma, let's use the Myhill-Nerode theorem to show that $L$ is regular.
  Suppose that $L$ is regular, then the theorem says that $\equiv_L$ has a finite number of equivalence classes.
  Therefore, we show that $\equiv_L$ actually has an infinite number of equivalence classes, a \emph{proof by contradiction}.

  Consider the string $0^n$ for $n \geq 0$.
  For each such string, there is a single extension in which $L$ contains the resulting string---$1^n$---and all other extensions result in strings not in $L$.
  Therefore $\equiv_L$ contains one equivalence class for each $n \geq 0$ corresponding to the starting string $0^n$.
  This is an infinite number of such classes, therefore $L$ is not regular.
\end{proof}

\end{document}
