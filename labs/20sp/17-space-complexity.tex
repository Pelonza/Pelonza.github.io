\documentclass[11pt]{book}

\setcounter{chapter}{17}
\newcommand{\docclass}{CSC 341 (20sp)}
\newcommand{\doctitle}{Space Complexity}
\newcommand{\docauthor}{Peter-Michael Osera}

\usepackage{reading}

\begin{document}

\begin{center}
  \large\textbf{{\doctitle}}
\end{center}

\vspace{2em}

%%%%% Problem-specific Macros %%%%%%%%%%%%%%%%%%%%%%%%%%%%%%%%%%%%%%%%%%%%%%%%%%

\newcommand{\desc}[1]{\ensuremath{\langle #1 \rangle}}
\newcommand{\NP}{\ensuremath{\mathsf{NP}}\xspace}
\newcommand{\prob}[1]{\ensuremath{\mathsf{#1}}\xspace}

%%%%%%%%%%%%%%%%%%%%%%%%%%%%%%%%%%%%%%%%%%%%%%%%%%%%%%%%%%%%%%%%%%%%%%%%%%%%%%%%

\begin{problem}{Space!}
  Prove that the following problems are in \( \text{PSPACE} \).
  \begin{align*}
    \prob{CLIQUE} =&\; \{ \desc{G, k} \mid \text{\( G \) has a \( k \)-clique} \} \\
    \prob{SUBSET} =&\; \{ \desc{S, t} \mid \text{There exists a \( S^* ⊆ S \) such that \( \sum_{x ∈ S^*} x = t \).} \}
  \end{align*}
\end{problem}

\newpage

\begin{problem}{The Space-Time Continuum}
  The book gives the following relationships between complexity classes that relate space and time:
    \[
      \text{P} ⊆ \text{NP} ⊆ \text{PSPACE} = \text{NPSPACE} ⊆ \text{EXPTIME}
    \]
    Where \( \text{PSPACE} = \text{NPSPACE} \) is proven with Savitch's Theorem which we will discuss next class.

    We also note that \( \text{P} \subset \text{EXPTIME} \).
    Thus at least one of the subset relations is proper.
    (The field generally believes that all of them are proper.)

\begin{enumerate}
  \item Review lemma 7.20 (pp. 294) that relates verifiers and nondeterministic polynomial time Turing machines.
    Prove the second of these relations: \( \text{NP} \subseteq \text{PSPACE} \).
  \item The book explains why \( \text{PSPACE} ⊆ \text{EXPTIME} \) by way of lemma 5.8 (pp. 222).
    Read this lemma and reformulate its proof to this claim which draws a correspondence between space and time.
\end{enumerate}

\end{problem}

\end{document}
