\documentclass[11pt]{book}

\setcounter{chapter}{11}
\newcommand{\docclass}{CSC 341 (20sp)}
\newcommand{\doctitle}{Properties and Variants}
\newcommand{\docauthor}{Peter-Michael Osera}

\usepackage{reading}

\begin{document}

\begin{center}
  \large\textbf{{\doctitle}}
\end{center}

\vspace{2em}

%%%%%%%%%%%%%%%%%%%%%%%%%%%%%%%%%%%%%%%%%%%%%%%%%%%%%%%%%%%%%%%%%%%%%%%%%%%%%%%%

\begin{problem}{Multi-(tape) Kill}
\begin{enumerate}[label=(\alph*)]
  \item Summarize each direction of the proof of equivalence of ordinary TMs and multi-tape Turing machines in a sentence a piece.
  \item Multi-tape Turing machines have a finite, fixed number of tapes at construction time.
    Suppose we tried extending the model so that the machine had an \emph{infinite} number of tapes at construction time.
    Describe the fundamental difficulties we would encounter in trying to construct and execute such a model.
  \item In contrast, imagine extending the model so that the multi-tape model can have a \emph{dynamic}, \emph{unbounded} number of tapes.
    The model would need an operation to spawn a new tape and the transition function would need to be modified to operate over a list of tapes.
    In contrast to the infinite tape model, does this variant seem equivalent to ordinary TMs?
    Why?
\end{enumerate}
\end{problem}

\vspace{1.5in}

\begin{problem}{The Matrix}
\begin{enumerate}[label=(\alph*)]
  \item Summarize each direction of the proof of equivalence of ordinary TMs and non-deterministic TMs in a sentence a piece.
  \item When converting NFAs to DFAs, we induced a \emph{state explosion}, drastically increasing the space complexity of the automata.
    Where does the \emph{space hit} occur in the NTM-to-TM transformation?
  \item Review the proof of the NTM-to-TM transformation in the book (theorem 3.16).
    (i) What is the purpose of the ``address'' in the construction?
    (ii) What does the TM do if a string is not accepted by the original NTM?
    How do we fix this problem?
\end{enumerate}
\end{problem}

\newpage

\begin{problem}{Blah Blah Blah}
\begin{enumerate}[label=(\alph*)]
  \item Summarize each direction of the proof of equivalence of ordinary TMs and enumerators in a sentence a piece.
  \item Give a formal description of \emph{what} an enumerator is.
    Include a rigorous description of its components, its behavior, and the inclusion condition for its language, \ie, the condition under which \( w \in L(E) \) for some enumerator \( E \)
\end{enumerate}
\end{problem}


\vspace{5in}

\noindent \textbf{Exam \#1 Topics}
\begin{enumerate}
  \item Machine descriptions (\eg, formal TM diagrams).
  \item Language closure (\eg, show that \( L_1 \circ L_2 \) is closed under the regular languages).
  \item Language non-inclusion (\eg, with the pumping lemmas or Myhill-Nerode).
  \item Machine analysis (\eg, emptiness of a DFA's language).
  \item Machine equivalence (\eg, show that 2-PDA is equivalento a TM).
\end{enumerate}

\end{document}
