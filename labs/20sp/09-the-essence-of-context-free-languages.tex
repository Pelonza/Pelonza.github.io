\documentclass[11pt]{book}

\setcounter{chapter}{9}
\newcommand{\docclass}{CSC 341 (20sp)}
\newcommand{\doctitle}{The Essence of Context-Free Languages}
\newcommand{\docauthor}{Peter-Michael Osera}

\usepackage{reading}

\begin{document}

\begin{center}
  \large\textbf{{\doctitle}}
\end{center}

\vspace{2em}

%%%%%%%%%%%%%%%%%%%%%%%%%%%%%%%%%%%%%%%%%%%%%%%%%%%%%%%%%%%%%%%%%%%%%%%%%%%%%%%%

\begin{problem}{The Price of Power}

Context-free languages are more expressive than regular languages.  In this
problem, we examine the trade-offs of that expressiveness, namely that certain
operations over regular languages are no longer closed under context-free
languages.

\begin{enumerate}[itemsep=0pt, label=(\alph*)]
  \item Prove that context-free languages are closed under:
    \begin{itemize}
      \item Union $L_1 \cup L_2$.
      \item Concatenation $L_1 \circ L_2$, and
      \item Kleene star $L^*$.
    \end{itemize}
    (\emph{Hint}: in all three cases, you should use context-free grammars
    as your model to show closure.)
  \item Both \( L_1 \) and \( L_2 \) below are context-free:
    \begin{gather*}
      L_1 = \{\, a^i b^j c^k \mid i < j \,\} \\
      L_2 = \{\, a^i b^j c^k \mid i < k \,\}
    \end{gather*}
    Give either a CFG or PDA for one of these languages, demonstrating that it is indeed context-free.
  \item Show that language intersection \( L_1 \cap L_2 \) is \emph{not} closed under context-free languages.
    (\emph{Hint}: what happens when you take the intersection of \( L_1 \) and \( L_2 \)?)
  \item Use the previous parts of this problem to conclude that language complement \( \overline{L} \) (the set of strings \emph{not} in \( L \)) is also \emph{not} closed under context-free languages.
    (\emph{Hint}: use standard set-theoretic operations that involve intersection and complement to push through a proof by contradiction.)
\end{enumerate}

\end{problem}

\end{document}
