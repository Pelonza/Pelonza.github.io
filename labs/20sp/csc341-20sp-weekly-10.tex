\documentclass[12pt]{article}

%%%%% Document Information (FILL THIS IN!) %%%%%
\newcommand{\authornames}{(Names)}

\newcommand{\coursenumber}{CSC 341}
\newcommand{\assignmentname}{Weekly Exercises \#10}
\newcommand{\assignmentversion}{2}

%%%%% Fonts and Encodings %%%%%%%%%%%%%%%%%%%%%%%%%%%%%%%%%%%%%%%%%%%%%%%%%%%%%%
\usepackage[T1]{fontenc}
\usepackage{libertine}

%%%%% General %%%%%%%%%%%%%%%%%%%%%%%%%%%%%%%%%%%%%%%%%%%%%%%%%%%%%%%%%%%%%%%%%%
\usepackage{amsmath}
\usepackage{amssymb}
\usepackage{enumitem}
\usepackage{framed}
\usepackage{fancyvrb}
\usepackage[amsmath, amsthm, thmmarks]{ntheorem}
\usepackage{microtype}
\usepackage{url}
\usepackage[svgnames]{xcolor}
\usepackage{xspace}

\usepackage{tikz}

%%%%% Page Formatting %%%%%%%%%%%%%%%%%%%%%%%%%%%%%%%%%%%%%%%%%%%%%%%%%%%%%%%%%%
\usepackage[
  top=1in,
  bottom=1in,
  left=1in,
  right=1in,
  includefoot,
  paperwidth=8.5in,
  paperheight=11in
]{geometry}

\usepackage{fancyhdr}
\setlength{\headheight}{15.2pt}
\pagestyle{fancy}

\fancyhead{}
\fancyfoot{}
\lhead{\coursenumber{}---\assignmentname{} (ver. \assignmentversion), \authornames{}}
\cfoot{\thepage}

%%%%% Code %%%%%%%%%%%%%%%%%%%%%%%%%%%%%%%%%%%%%%%%%%%%%%%%%%%%%%%%%%%%%%%%%%%%%

\usepackage{listings}
\usepackage{xcolor}

% Adapted from: https://en.wikibooks.org/wiki/LaTeX/Source_Code_Listings
\definecolor{mygreen}{rgb}{0,0.6,0}
\definecolor{mygray}{rgb}{0.5,0.5,0.5}
\definecolor{mymauve}{rgb}{0.58,0,0.82}

\lstset{ 
  backgroundcolor=\color{white},      % choose the background color; you must add \usepackage{color} or \usepackage{xcolor}; should come as last argument
  basicstyle=\ttfamily\footnotesize,  % the size of the fonts that are used for the code
  breakatwhitespace=false,            % sets if automatic breaks should only happen at whitespace
  breaklines=true,                    % sets automatic line breaking
  captionpos=b,                       % sets the caption-position to bottom
  commentstyle=\color{mygreen},       % comment style
  keepspaces=true,                    % keeps spaces in text, useful for keeping indentation of code (possibly needs columns=flexible)
  keywordstyle=\color{blue},          % keyword style
  language=Java,                      % the language of the code
  stringstyle=\color{mymauve},        % string literal style
}

%%%%% Basic Macros and Definitions %%%%%%%%%%%%%%%%%%%%%%%%%%%%%%%%%%%%%%%%%%%%%
\newtheorem{claim}{Claim}
\newtheorem{invariant}{Invariant}
\newtheorem{defn}{Definition}
\newtheorem{thm}{Theorem}
\newtheorem{lemma}{Lemma}

\newcommand{\ie}{\emph{i.e.}\xspace}
\newcommand{\eg}{\emph{e.g.}\xspace}
\newcommand{\hint}[1]{(\emph{Hint}: #1)}
\newcommand{\note}[1]{(\emph{Note}: #1)}

\definecolor{shadecolor}{named}{LightGray}
\newenvironment{exercise}[1]{%
  \begin{shaded}
  \noindent\textbf{Exercise (#1)}\quad
}{\end{shaded}}

\newcounter{ProblemCounter}
\newenvironment{problem}[1][]
  {\refstepcounter{ProblemCounter}\noindent\textbf{Problem \theProblemCounter{} (#1)}\quad}
  {\newpage}

\newcommand{\answerbelow}{\noindent\makebox[\linewidth]{\rule{\textwidth}{0.4pt}}}

%%%%% Problem-specific Macros %%%%%%%%%%%%%%%%%%%%%%%%%%%%%%%%%%%%%%%%%%%%%%%%%%

\newcommand{\prob}[1]{\ensuremath{\textsf{#1}}\xspace}
\newcommand{\desc}[1]{\ensuremath{\langle #1 \rangle}}
\newcommand{\comp}[1]{\ensuremath{\overline{#1}}\xspace}

\newcommand{\nats}{\ensuremath{\mathbb{N}}\xspace}
\newcommand{\dfa}{\textsf{DFA}\xspace}
\newcommand{\EQdfa}{\ensuremath{EQ_\mathsf{DFA}\xspace}}
\newcommand{\Atm}{\ensuremath{A_\mathsf{TM}}\xspace}

%%%%%%%%%%%%%%%%%%%%%%%%%%%%%%%%%%%%%%%%%%%%%%%%%%%%%%%%%%%%%%%%%%%%%%%%%%%%%%%%

\begin{document}

%%%%% Monday: Undecidability Proof Techniques %%%%%%%%%%%%%%%%%%%%%%%%%%%%%%%%%%

\begin{problem}[Enter the Machine]

Many problems concerning properties of models of computations turn out to be
decidable.  We can break this up into two sorts:
\begin{enumerate}
  \item \emph{Language properties} that concern the strings that a machine
    accepts.
  \item \emph{Behavioral properties} that concern how a machine operates.
\end{enumerate}
In this problem, we will discuss properties of the second sort.  Because these
properties are not tied to the acceptance/rejection behavior of the machine,
undecidablity proofs of these properties all follow a strict formula.

Let's consider such a problem from the textbook:
\begin{quote}
  (Sipser 5.13) A useless state in a Turing machine is one that is never
  entered on any input string.  Consider the problem of determining whether a
  Turing machine has any useless states.  Formulate this problem as a language
  and show it is undecidable.
\end{quote}

\begin{exercise}{Languages}
  Do what Sipser says!  First formulate this problem as a language
  \prob{USELESS}.  Importantly, identify what an assumed decider \( D \) for
  \prob{USELESS} takes as input.
\end{exercise}

Now let's show that \prob{USELESS} is undecidable by a mapping reduction from
\Atm.  To do so, we'll follow the strategy suggested in the handout that
accompanied today's reading.  First let's establish what our mapping function
should look like and how it should behave.

\begin{exercise}{Mappings}
  The heart of the mapping reduction is a mapping from inputs to \Atm to
  \prob{USELESS}.  What does the mapping function \( f \) take as input and
  produces as output?
\end{exercise}

Once we know the types of the mapping reduction, we now need to reason about
the correctness condition linking acceptance between \Atm and \prob{USELESS}.

\begin{exercise}{The Condition}
  Let \( \desc{M, w} \) be inputs to a decider for \Atm and \( \desc{M'} \) the
  input to a decider for \prob{USELESS}.  Give the correctness condition for \(
  f \) based on these inputs.
\end{exercise}

We have now established our correctness condition for our construction.  Based
on this condition, we need to figure out a way to have our machine \( M' \)
\emph{conditionally} have the property of \prob{USELESS}: \( M' \) has a useless
state.  Recall that a state is useless if no input string causes \( M '\) to
enter that state.  Because the transitions of a Turing machine are dependent on
the positions of the tape head and contents of the tape at that position, we
need to know how a Turing machine executes in order to determine if it contains
a useless state.  This is ultimately why \Atm can be reduced to \prob{USELESS}.

For the purposes of our construction, however, we want to conditionally make \(
M' \) have a useless state in such a way that it is painfully obvious that this
is the case.  This will simplify our reasoning when we go to make sure that \(
M' \) obeys the correct condition outlined above.  Two ways that we might go
about this are:
\begin{enumerate}
  \item Give \( M' \) a state \( u \) that has no physical transition to it from
    any other state in \( M' \).  It is obvious that \( u \) is useless because
    there is no way to reach it in this situation.
  \item Give \( M' \) a state \( u \) that is connected from some other state in
    the Turing machine, but make that transition involve reading a tape symbol
    that provably never appears on the tape, \eg, by ensuring that it is a
    alphabet symbol and it is not mentioned in the read position of any
    transition of the Turing Machine.
\end{enumerate}
The first option does not not work for our purposes because the transitions are
fixed when \( M' \) is constructed; we can't condition the existence of a
transition based on whether \( M \) accepts \( w \).  However, the second option
is controllable at runtime---we can condition the appearance of the symbol that
will allow us to transition to our useless state on whether \( M \) accepts \( w
\).

Putting all of this together, we have the following skeleton for the
construction of our mapping function and the Turing machine \( M' \) it
produces:

\vspace{1em}

\noindent \( M' = \text{``On input \( x \):} \)
\begin{enumerate}[topsep=0pt, itemsep=0pt]
  \item Run \( M \) on \( w \).
    \begin{itemize}[topsep=0pt, itemsep=0pt]
      \item \( M \) accepts: \emph{(\( M' \) should have a useless state)}
      \item \( M \) rejects: \emph{(\( M' \) should not have a useless state)}''
    \end{itemize}
\end{enumerate}

Note that also in the case where \( M \) loops on \( w \), we want to have the
same behavior as the rejection case.

\begin{exercise}{The Construction}
  Complete the construction of \( M' \) as specified above.
\end{exercise}

Finally, we need to verify that \( M' \) does the right thing.  Rather than
proving our correctness condition instead (which is a biconditional), I find it
easier to reason about the three cases that might occur as a result of \( M \)'s
execution of \( w \) (convince yourself these three cases cover the two cases of
the biconditional you derived above):
\begin{itemize}[itemsep=0pt]
  \item \( M \) accept \( w \).
  \item \( M \) rejects \( w \).
  \item \( M \) loops on \( w \).
\end{itemize}

\begin{exercise}{The Proof of Correctness}
  Prove that \( M' \) has the appropriate behavior for the three cases outlined
  above with respect to your correctness condition for the mapping function.
\end{exercise}

\answerbelow
% FILL IN YOUR ANSWER HERE

\end{problem}

%%%%% Friday: Rice's Theorem %%%%%%%%%%%%%%%%%%%%%%%%%%%%%%%%%%%%%%%%%%%%%%%%%%%

\begin{problem}[Rice's Theorem]

In this problem, we'll walk through and complete a proof of Rice's Theorem
which was introduced in the reading.

Let \( P \)  be a decidable language whose strings are turing machine
descriptions that satisfy an arbitrary predicate \( f \), \ie,
\[
  P = \{ \desc{M} \mid f(M) \}
\]
where \( f \) is a proposition ranging over Turing machine descriptions  For
example, the following are propositions over a Turing machine, call it \( M \):
\begin{itemize}
  \item \( L(M) \) is finite.
  \item \( M \) writes a `\$' on its tape for some input.
\end{itemize}

Rice's Theorem demands two properties are true about \( P \).
\begin{enumerate}
  \item \( P \) is non-trivial: \( |P| > 0 \) and \( P \neq \Sigma^* \).
  \item \( P \) is a language property: if \( L(M_1) = L(M_2) \) then \(
    \desc{M_1} \in P \Leftrightarrow \desc{M_2} \in P \).
\end{enumerate}

Let's first check that we understand what we mean by language property.

\begin{exercise}{Language Property Example}
  Let:
  \[
    P = \{ \desc{M} \mid \text{\( M \) writes a `\$' on its tape for some
    input.} \}.
  \]
  Spend 2--3 minutes individually and write down why \( P \) does not fulfill
  the conditions of a language property.  Come back as a group and come up with
  a final explanation why \( P \) is not a language property.
\end{exercise}

Now let's proceed with the core of the proof of Rice's theorem.  First we'll
complete the core reduction under a set of assumptions about our property \( P
\).  We'll then discuss after the fact why these assumptions are sound given the
constraints of Rice's theorem.  Assume the following:

\begin{itemize}
  \item There exists a TM \( T \) such that \( \desc{T} \in P \).
  \item Define the TM that rejects all strings be \( T_{\emptyset} \).
    Assume that \( \desc{T_{\emptyset}} \not\in P \).
\end{itemize}

\begin{exercise}{Core Reduction}
  As a group, use \( T \) to construct a mapping reduction from \( \Atm \) to \(
  P \).  Keep in mind the following insights while constructing your mapping
  reduction:
  \begin{itemize}
    \item Your mapping construction, when given an input to \Atm of the form
      \( \desc{M, w} \) should produce a TM \( \desc{M'} \) with the property
      that \( M \) accepts \( w \) iff \( \desc{M'} \in P \).
    \item Note that we have asserted that (a) a machine \( T \) exists that is
      in \( P \) and (b) the machine that rejects all strings \( T_\emptyset \)
      is not in \( P \).  Use these as the targets for your mapping function as
      described in ``Undecidable Proof Strategies'' handout.
  \end{itemize}
  Once you have a candidate reduction, check your answer with the course staff.
\end{exercise}

With the reduction completed, we need to show where the two constraints came
from.  Let's briefly talk about the first constraint.

\begin{exercise}{The First Constraint}
  Individually, spend a few minutes and explain why the constraints of Rice's
  Theorem on \( P \) imply that there exists a TM \( T \) such that \( \desc{T}
  \in P \).  Come back as a group and agree on a final explanation.
\end{exercise}

The second constraint is really an assumption ``without loss of generality''.
In other words, even if \( T_\emptyset \in P \), we can still complete the
proof with a little bit inconsequential fix-up to our proof.  Note that we
can't simply work with \( T_\emptyset \) directly in this case because our
mapping reduction will fail in a style similar to \( E_{TM} \).  We instead
need to do something else.

\begin{exercise}{The Final Fix-Up}
  It turns out that the fix-up is to consider instead the complement of \( P \):
  \[
    \overline{P} = \{\, \desc{M} \mid \text{\( f(M) \) is not provable} \,\}.
  \]
  The reduction above then holds, but shows that \( \Atm \leq_m \overline{P} \)
  rather than \( P \). As a group, determine how a decider for \( \overline{P}
  \) could be used to create a decider for \( P \), allowing us to conclude
  that \( \Atm \leq \overline{P} \) as well.  Check your reasoning for this and
  the previous part with the course staff when you are done.
\end{exercise}

\end{problem}


\end{document}
