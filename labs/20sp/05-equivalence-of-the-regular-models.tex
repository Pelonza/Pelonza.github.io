\documentclass[11pt]{book}

\setcounter{chapter}{5}
\newcommand{\docclass}{CSC 341 (20sp)}
\newcommand{\doctitle}{Equivalence of the Regular Models}
\newcommand{\docauthor}{Peter-Michael Osera}

\usepackage{reading}

\begin{document}

\begin{center}
  (Turn-in problems are indicated with a dagger \turninproblem{}.)
\end{center}

\begin{center}
  \large\textbf{{\doctitle}}
\end{center}

\vspace{2em}

%%%%%%%%%%%%%%%%%%%%%%%%%%%%%%%%%%%%%%%%%%%%%%%%%%%%%%%%%%%%%%%%%%%%%%%%%%%%%%%%

\begin{problem}{Back-and-Forth}

We have covered three models of computation---DFAs, NFAs, and regular expressions---and have shown that all three models are equivalent, \ie, they recognize the class of regular languages.
Specifically, we show the following equivalences:
\[
  \mathsf{DFA} \Leftrightarrow \mathsf{NFA} \Leftrightarrow \mathsf{RegEx}
\]
Give a proof sketch for each of the four implications indicated above.

\end{problem}

\newpage

\begin{problem}{Compilation}

Consider the regular expression \( R = 0^* 11 0^* \).
Use the proofs of equivalence of the regular language models to (a) convert \( R \) to a NFA \( N \) and then (b) convert \( N \) to a DFA \( D \).

\end{problem}

\end{document}
