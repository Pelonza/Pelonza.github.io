\documentclass[12pt]{article}

%%%%% Document Information (FILL THIS IN!) %%%%%
\newcommand{\authornames}{(Names)}

\newcommand{\coursenumber}{CSC 341}
\newcommand{\assignmentname}{Weekly Exercises \#9}
\newcommand{\assignmentversion}{2}

%%%%% Fonts and Encodings %%%%%%%%%%%%%%%%%%%%%%%%%%%%%%%%%%%%%%%%%%%%%%%%%%%%%%
\usepackage[T1]{fontenc}
\usepackage{libertine}

%%%%% General %%%%%%%%%%%%%%%%%%%%%%%%%%%%%%%%%%%%%%%%%%%%%%%%%%%%%%%%%%%%%%%%%%
\usepackage{amsmath}
\usepackage{amssymb}
\usepackage{enumitem}
\usepackage{fancyvrb}
\usepackage[amsmath, amsthm, thmmarks]{ntheorem}
\usepackage{microtype}
\usepackage{url}
\usepackage[svgnames]{xcolor}
\usepackage{xspace}

\usepackage{tikz}

%%%%% Page Formatting %%%%%%%%%%%%%%%%%%%%%%%%%%%%%%%%%%%%%%%%%%%%%%%%%%%%%%%%%%
\usepackage[
  top=1in,
  bottom=1in,
  left=1in,
  right=1in,
  includefoot,
  paperwidth=8.5in,
  paperheight=11in
]{geometry}

\usepackage{fancyhdr}
\setlength{\headheight}{15.2pt}
\pagestyle{fancy}

\fancyhead{}
\fancyfoot{}
\lhead{\coursenumber{}---\assignmentname{} (ver. \assignmentversion), \authornames{}}
\cfoot{\thepage}

%%%%% Code %%%%%%%%%%%%%%%%%%%%%%%%%%%%%%%%%%%%%%%%%%%%%%%%%%%%%%%%%%%%%%%%%%%%%

\usepackage{listings}
\usepackage{xcolor}

% Adapted from: https://en.wikibooks.org/wiki/LaTeX/Source_Code_Listings
\definecolor{mygreen}{rgb}{0,0.6,0}
\definecolor{mygray}{rgb}{0.5,0.5,0.5}
\definecolor{mymauve}{rgb}{0.58,0,0.82}

\lstset{ 
  backgroundcolor=\color{white},      % choose the background color; you must add \usepackage{color} or \usepackage{xcolor}; should come as last argument
  basicstyle=\ttfamily\footnotesize,  % the size of the fonts that are used for the code
  breakatwhitespace=false,            % sets if automatic breaks should only happen at whitespace
  breaklines=true,                    % sets automatic line breaking
  captionpos=b,                       % sets the caption-position to bottom
  commentstyle=\color{mygreen},       % comment style
  keepspaces=true,                    % keeps spaces in text, useful for keeping indentation of code (possibly needs columns=flexible)
  keywordstyle=\color{blue},          % keyword style
  language=Java,                      % the language of the code
  stringstyle=\color{mymauve},        % string literal style
}

%%%%% Basic Macros and Definitions %%%%%%%%%%%%%%%%%%%%%%%%%%%%%%%%%%%%%%%%%%%%%
\newtheorem{claim}{Claim}
\newtheorem{invariant}{Invariant}
\newtheorem{defn}{Definition}
\newtheorem{thm}{Theorem}
\newtheorem{lemma}{Lemma}

\newcommand{\ie}{\emph{i.e.}\xspace}
\newcommand{\eg}{\emph{e.g.}\xspace}
\newcommand{\hint}[1]{(\emph{Hint}: #1)}
\newcommand{\note}[1]{(\emph{Note}: #1)}

\newcounter{ProblemCounter}
\newenvironment{problem}[1][]
  {\refstepcounter{ProblemCounter}\noindent\textbf{Problem \theProblemCounter{} (#1)}\quad}
  {\newpage}

\newcommand{\answerbelow}{\noindent\makebox[\linewidth]{\rule{\textwidth}{0.4pt}}}

%%%%% Problem-specific Macros %%%%%%%%%%%%%%%%%%%%%%%%%%%%%%%%%%%%%%%%%%%%%%%%%%

\newcommand{\prob}[1]{\ensuremath{\textsf{#1}}\xspace}
\newcommand{\desc}[1]{\ensuremath{\langle #1 \rangle}}
\newcommand{\comp}[1]{\ensuremath{\overline{#1}}\xspace}

\newcommand{\nats}{\ensuremath{\mathbb{N}}\xspace}
\newcommand{\dfa}{\textsf{DFA}\xspace}
\newcommand{\EQdfa}{\ensuremath{EQ_\mathsf{DFA}\xspace}}
\newcommand{\Atm}{\ensuremath{A_\mathsf{TM}}\xspace}

%%%%%%%%%%%%%%%%%%%%%%%%%%%%%%%%%%%%%%%%%%%%%%%%%%%%%%%%%%%%%%%%%%%%%%%%%%%%%%%%

\begin{document}

%%%%% Monday: Machine Analysis %%%%%%%%%%%%%%%%%%%%%%%%%%%%%%%%%%%%%%%%%%%%%%%%%

\begin{problem}[To Infinity And Beyond]

\begin{center}
\note{You only need to do \textbf{one} of problems 1 or 2 for the weekly.}
\end{center}

Let \( \EQdfa = \{\, \desc{D_1, D_2} \mid \text{\( D_1 \) and \( D_2 \) are
equivalent DFAs} \,\} \).  Recall that two DFAs are equivalent if their
languages are identical.  Theorem 4.5 of Sipser shows that \( \EQdfa \) is
decidable by way of language closure properties.  In this problem, we'll
consider a more direct approach.

\begin{enumerate}[label=(\alph*)]
  \item Consider the following proof that \( \EQdfa \) is decidable by way of
    constructing a deciding Turing machine that recognizes the language.
    \begin{proof}
      Define \( M \), a deciding Turing machine that decides \( \EQdfa \) as
      follows:

      \noindent \( M = \text{``On input \( \desc{D_1, D_2} \)---two DFAs:} \)
      \begin{enumerate}
        \item For every possible string \( w \in \Sigma^* \), if \( D_1 \) and
          \( D_2 \) have differing acceptance behavior on \( w \), \ie, one
          rejects and the other accepts, then \emph{reject}.
        \item Otherwise, \( D_1 \) and \( D_2 \) have the same acceptance
          behavior on all strings: \emph{accept}''.
        \end{enumerate}
    \end{proof}
    This construction has a fatal flaw!  In a few sentences, describe what the
    flaw is.
  \item We can patch up this flaw by noting that we don't have to test all
    strings!  Determine a bound on the size of strings we need to test to
    determine if two DFAs are equal.  In a few sentences, argue why this bound
    is correct.
\end{enumerate}

\end{problem}

\begin{problem}[Can You Do It?  Yes You Can!]

\begin{center}
  \note{You only need to do \textbf{one} of problems 1 or 2 for the weekly.}
\end{center}

\begin{enumerate}[label=(\alph*)]
  \item (\emph{Optional}).  Call a language \( L \) \emph{prefix-free} if for
    all \( w \in L \), there does not exist a \( w' \in L \) such that \( w
    \neq w' \) and \( w' \) is a prefix of \( w \). For example \( ab \) is a
    prefix of \( abcde \) and thus \( abcde \) would not be in a prefix-free
    language if \( ab \) was also in the language.  Give a deciding procedure
    to determine if the language of a DFA \( D \) is prefix free. Prove that it
    is correct by arguing both correctness and termination of your algorithm.
  \item Apply the construction you developed to the problem of determining
    whether a pushdown automata (PDA) is prefix-free.  Identify:
    \begin{itemize}
      \item What you must change about the construction in order to adapt it
        to PDAs.
      \item What problems do you ultimately run into with your construction
      that you cannot reconcile?
    \end{itemize}
  \item The previous part does not necessarily mean that the prefix-free
    property is not decidable for PDAs in general.  Individually, spend
    approximately 10 minutes brainstorming an alternative approach to
    determining whether a PDA is prefix-free.  Come back and determine which of
    your algorithms has the most promise.  Describe the algorithm at a
    high-level and give a positive example of its execution, \ie, a PDA whose
    language is prefix-free and how the algorithm accepts this PDA.
  \item It turns out that the problem of determining whether a PDA is prefix
    free is actually undecidable! Since we can't prove this fact yet, give an
    example of a PDA that your algorithm should accept or reject, but the
    algorithm either fails to give the correct answer or goes into an infinite
    loop.
\end{enumerate}

\answerbelow{}

% FILL IN YOUR ANSWER HERE

\end{problem}

%%%%% Wednesday: Diagonalization %%%%%%%%%%%%%%%%%%%%%%%%%%%%%%%%%%%%%%%%%%%%%%%

\begin{problem}[Countability]

\begin{center}
  \note{This problem is \textbf{not required} for the weekly turn-in; we
  discussed it in class!}
\end{center}

\begin{enumerate}[label=(\alph*)]
  \item Prove that the language \( L = \Sigma^* \) where \( \Sigma = \{ 0, 1 \}
    \) is countably infinite.  \hint{simply declaring that the mapping from \(
    \Sigma^* \) to \( \mathbb{N} \) is the binary interpretation of \( w \in
    \Sigma^* \) is insufficient since \( 0 \) and \( 000 \) are distinct
    strings but both represent the value zero.  The resulting mapping would,
    therefore, not be a bijection!}
  \item Generalize your construction to any \( \Sigma \) of finite size.
  \item Use this fact to argue that the set of possible Java programs is
    countably infinite.
\end{enumerate}

\answerbelow{}

% FILL IN YOUR ANSWER HERE

\end{problem}

\begin{problem}[Hey, Wait a Second]

Consider the following proof that \( \nats \) is uncountable.

\begin{proof}
  Assume that \( \nats \) is countable. Then there is a bijection \( f \) that
  covers every natural numbers in \( \nats \). Construct the natural number \(
  n \) where the \( i \)th digit of \( n \) is the \( i \)th digit of the \( i
  \)th natural number in the bijection (\ie, \( f(i) \)) plus one mod 10 (so
  that it is a decimal digit). That is, if \( k \) is the \( i \)th digit of
  the \( i \)th natural number, then the \( i \)th digit of \( n \) is given by
  \( k + 1 \mod 10 \). \( n \) is a valid natural number and by construction,
  \( n \) differs from every natural number in the bijection by one digit.
  Therefore, \( n \) cannot be in the bijection and therefore our assumption
  that such a bijection exists is incorrect.  Thus, \( \nats \) is uncountable.
\end{proof}

However, we already know that \( \nats \) is countable (the bijection is the
identity function).  What is wrong with this proof? \hint{think about the
assumptions latent in Cantor's diagonalization argument that the reals are
uncountable.  What specifically is different in this case?}

\answerbelow{}

% FILL IN YOUR ANSWER HERE

\end{problem}

%%%%% Friday: Undecidability %%%%%%%%%%%%%%%%%%%%%%%%%%%%%%%%%%%%%%%%%%%%%%%%%%%

\begin{problem}[It Hurts My Head]

Prove that the following languages are undecidable by reduction from a known
undecidable language.
\begin{enumerate}[label=(\alph*)]
  \item \( L_1 = \{ \desc{M} \mid \text{\( M \) is a TM that accepts \( w^{R}
    \) whenever it accepts \( w \)} \} \).  (Recall that \( w^{R} \) is the
    reversal of \( w \).)
  \item \( L_2 = \{ \desc{M, w} \mid \text{\( M \) is a TM that, on input \( w
    \), writes a '\$' on the tape} \} \).
\end{enumerate}

\answerbelow{}

% FILL IN YOUR ANSWER HERE

\end{problem}

%%%%%%%%%%%%%%%%%%%%%%%%%%%%%%%%%%%%%%%%%%%%%%%%%%%%%%%%%%%%%%%%%%%%%%%%%%%%%%%%

\end{document}
