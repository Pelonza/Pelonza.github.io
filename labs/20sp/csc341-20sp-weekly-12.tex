\documentclass[12pt]{article}

%%%%% Document Information (FILL THIS IN!) %%%%%
\newcommand{\authornames}{(Names)}

\newcommand{\coursenumber}{CSC 341}
\newcommand{\assignmentname}{Weekly Exercises \#12}
\newcommand{\assignmentversion}{1}

%%%%% Fonts and Encodings %%%%%%%%%%%%%%%%%%%%%%%%%%%%%%%%%%%%%%%%%%%%%%%%%%%%%%
\usepackage[T1]{fontenc}
\usepackage{libertine}

%%%%% General %%%%%%%%%%%%%%%%%%%%%%%%%%%%%%%%%%%%%%%%%%%%%%%%%%%%%%%%%%%%%%%%%%
\usepackage{amsmath}
\usepackage{amssymb}
\usepackage[shortlabels]{enumitem}
\setlist{nosep}
\usepackage{framed}
\usepackage{fancyvrb}
\usepackage[amsmath, amsthm, thmmarks]{ntheorem}
\usepackage{microtype}
\usepackage{url}
\usepackage[svgnames]{xcolor}
\usepackage{xspace}

\usepackage{tikz}

%%%%% Page Formatting %%%%%%%%%%%%%%%%%%%%%%%%%%%%%%%%%%%%%%%%%%%%%%%%%%%%%%%%%%
\usepackage[
  top=1in,
  bottom=1in,
  left=1in,
  right=1in,
  includefoot,
  paperwidth=8.5in,
  paperheight=11in
]{geometry}

\usepackage{fancyhdr}
\setlength{\headheight}{15.2pt}
\pagestyle{fancy}

\fancyhead{}
\fancyfoot{}
\lhead{\coursenumber{}---\assignmentname{} (ver. \assignmentversion), \authornames{}}
\cfoot{\thepage}

%%%%% Code %%%%%%%%%%%%%%%%%%%%%%%%%%%%%%%%%%%%%%%%%%%%%%%%%%%%%%%%%%%%%%%%%%%%%

\usepackage{listings}
\usepackage{xcolor}

% Adapted from: https://en.wikibooks.org/wiki/LaTeX/Source_Code_Listings
\definecolor{mygreen}{rgb}{0,0.6,0}
\definecolor{mygray}{rgb}{0.5,0.5,0.5}
\definecolor{mymauve}{rgb}{0.58,0,0.82}

\lstset{ 
  backgroundcolor=\color{white},      % choose the background color; you must add \usepackage{color} or \usepackage{xcolor}; should come as last argument
  basicstyle=\ttfamily\footnotesize,  % the size of the fonts that are used for the code
  breakatwhitespace=false,            % sets if automatic breaks should only happen at whitespace
  breaklines=true,                    % sets automatic line breaking
  captionpos=b,                       % sets the caption-position to bottom
  commentstyle=\color{mygreen},       % comment style
  keepspaces=true,                    % keeps spaces in text, useful for keeping indentation of code (possibly needs columns=flexible)
  keywordstyle=\color{blue},          % keyword style
  language=Python,                    % the language of the code
  stringstyle=\color{mymauve},        % string literal style
}

%%%%% Basic Macros and Definitions %%%%%%%%%%%%%%%%%%%%%%%%%%%%%%%%%%%%%%%%%%%%%
\newtheorem{claim}{Claim}
\newtheorem{invariant}{Invariant}
\newtheorem{defn}{Definition}
\newtheorem{thm}{Theorem}
\newtheorem{lemma}{Lemma}

\newcommand{\ie}{\emph{i.e.}\xspace}
\newcommand{\eg}{\emph{e.g.}\xspace}
\newcommand{\hint}[1]{(\emph{Hint}: #1)}
\newcommand{\note}[1]{(\emph{Note}: #1)}

\newcounter{ProblemCounter}
\newenvironment{problem}[1][]
  {\refstepcounter{ProblemCounter}\noindent\textbf{Problem \theProblemCounter{} (#1)}\quad}
  {\newpage}

\newcommand{\answerbelow}{\noindent\makebox[\linewidth]{\rule{\textwidth}{0.4pt}}}

%%%%% Problem-specific Macros %%%%%%%%%%%%%%%%%%%%%%%%%%%%%%%%%%%%%%%%%%%%%%%%%%

\newcommand{\prob}[1]{\ensuremath{\textsf{#1}}\xspace}
\newcommand{\desc}[1]{\ensuremath{\langle #1 \rangle}}
\newcommand{\comp}[1]{\ensuremath{\overline{#1}}\xspace}

\newcommand{\nats}{\ensuremath{\mathbb{N}}\xspace}
\newcommand{\dfa}{\textsf{DFA}\xspace}
\newcommand{\EQdfa}{\ensuremath{EQ_\mathsf{DFA}\xspace}}
\newcommand{\Atm}{\ensuremath{A_\mathsf{TM}}\xspace}

%%%%%%%%%%%%%%%%%%%%%%%%%%%%%%%%%%%%%%%%%%%%%%%%%%%%%%%%%%%%%%%%%%%%%%%%%%%%%%%%

\begin{document}

%%%%% Monday: Approximation %%%%%%%%%%%%%%%%%%%%%%%%%%%%%%%%%%%%%%%%%%%%%%%%%%%%

\begin{problem}[Formation]

Before you begin the group activity for this week, identify:
\begin{itemize}
  \item A \emph{Scribe}: who will be responsible for creating a shared document
    and turning in your solutions for this week's problems.
  \item A \emph{Timekeeper}: who will keep track of the timer for weekly
    activities that require timekeeping.
  \item A \emph{Driver}: who will be responsible for asking people to share or
    otherwise kickstarting conversation if group chat slows down.
\end{itemize}

For this problem, simply identify your teammate's roles in the space below.

If you cannot attend a given class period, please work with your group on how
you will contribute to the group's work either asynchronously (\eg, by doing
the work yourself) or outside of the class period (\eg, by chatting or meeting
outside of class).  If there are any issues in coordinating work in your group,
please let me know as soon as possible!

\answerbelow
% FILL IN YOUR ANSWER HERE

\end{problem}

%%%%%%%%%%%%%%%%%%%%%%%%%%%%%%%%%%%%%%%%%%%%%%%%%%%%%%%%%%%%%%%%%%%%%%%%%%%%%%%%

\begin{problem}[Tours, Approximately]

A polynomial-time approximation scheme (PTAS) is an approximation algorithm for
a minimization problem that produces solutions that are within a factor of \(
(1 + \epsilon) \) of the optimal solution, for some constant parameter \(
\epsilon \).  In the book, Sipser describes a 2-approximation algorithm for
\prob{VERTEX-COVER} which is a PTAS with \( \epsilon = 1 \).  Several
NP-complete problems has PTAS.

Recall that a \emph{Hamiltonian cycle} is a cycle in a graph that touches every
vertex at least once.  From this, we can discuss the \emph{traveling salesman
problem} (TSP) which requires us to find the shortest Hamiltonian cycle in a
weighted undirected graph.  For simplicity's sake, when we talk about TSP, we
assume that the graph is complete, \ie, every location is reachable from every
other location.  It turns out that it is provable that TSP has no PTAS which is
not good since TSP occurs in many real-life situations!

In order to gain a tractable approximation for TSP, we need to add in
an extra constraint, \emph{the triangle inequality}, which states that for any
triple of vertices \( u \), \( v \), \( w \):
\[
  l(u, v) + l(v, w) \geq l(u, w)
\]
where \( l(u, v) \) is the weight of the edge \( uv \).  If we add this constraint
to TSP, then we can obtain a 2-approximation scheme for TSP.

\begin{enumerate}[(a)]
  \item Individually, spend no more than 2 minutes making sure you understand
    what the triangle inequality says.  Come up with a small example graph that
    exhibits the triangle inequality and one that does not.  Come back with
    your group and answer the following question: if we know that the triangle
    inequality holds of a graph, what do we know about the shortest path
    between any pair of vertices in the graph?
  \item As a group, spend no more than 3 minutes reviewing the concept of a
    \emph{minimum spanning tree} (MST) for a graph.  In particular, what is a MST
    and how can you compute one efficiently?  You do not need to include anything in
    your write-up for this part.
  \item Now, spend 10 minutes individually taking this idea of a MST and
    deriving an approximation algorithm for the TSP.  Come back as a group and agree
    on a final algorithm and describe it in your write-up.  \hint{Think about
    using the MST of the graph as a starting point for the tour?  How can the
    MST be used to visit all the vertices of the graph?  From this, you can view the
    problem as "fixing up" this tour so it is indeed Hamiltonian.}
  \item Finally, spend another 5 minutes individually to think about why this algorithm
    is a 2-approximation scheme.  That is, the result is no larger than two times the
    optimal tour of the graph.  Come back as a group and summarize your reasoning.
    You do not need to formally prove this fact, but instead try to give intuition why
    the factor is specifically 2 with respect to the triangle inequality and MSTs.
\end{enumerate}

\answerbelow
% FILL IN YOUR ANSWER HERE

\end{problem}

%%%%% Wednesday: Probabilistic Algorithms %%%%%%%%%%%%%%%%%%%%%%%%%%%%%%%%%%%%%%

\begin{problem}[Maybe?]

Consider the problem of analyzing a large data set \( S \) for the presence of
a particular value \( v \).  Here are two probabilistic algorithms that
describe how we might solve this problem.

\begin{lstlisting}
def find1(S, v):
    repeat forever:
        Choose an element x of S at random.
        If s == v, return true, else continue.
\end{lstlisting}

\begin{lstlisting}
def find2(S, v, k):
    for i in range(0, k):
        Choose an element x of S at random.
        If s == v, return true, else continue.
    # Otherwise the element was not found
    return false
\end{lstlisting}

\answerbelow
% FILL IN YOUR ANSWER HERE

\end{problem}

%%%%% Friday: Cryptography %%%%%%%%%%%%%%%%%%%%%%%%%%%%%%%%%%%%%%%%%%%%%%%%%%%%%

\begin{problem}[Forthcoming!]

\answerbelow
% FILL IN YOUR ANSWER HERE

\end{problem}

%%%%%%%%%%%%%%%%%%%%%%%%%%%%%%%%%%%%%%%%%%%%%%%%%%%%%%%%%%%%%%%%%%%%%%%%%%%%%%%%

\end{document}
